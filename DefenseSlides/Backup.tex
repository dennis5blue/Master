\begin{frame}[noframenumbering]{Backup: P-Frame Association Method}
\begin{itemize}
	\myItem Based on $diag(\mathbf{X})$, we can know the set of I-frame cameras and P-frame cameras
	\begin{itemize}
		\mySubItem Each P-frame camera can find its most correlated I-frame camera in linear time
		\mySubItem The remaining elements of $\mathbf{X}$ can be determined through $diag(\mathbf{X})$
	\end{itemize}
	\myItem An example of cameras schedule is shown below:
	\begin{figure}
		\centering
		\includegraphics[width=0.45\textwidth]{Figures/PscheOverview.pdf}
	\end{figure}
\end{itemize}
\end{frame}
%%
\begin{frame}[noframenumbering]{Backup: Correlation Re-Estimation Criterion}
\begin{columns}
\column{0.55\textwidth}
\begin{itemize}
	\myItem Denote $\eta_i^k$ as the improvement ratio of camera $i$ at transmission round $k$
	\myItem Denote $\bar{\eta}$ as the long term average improvement ratio
	\begin{itemize}
		\mySubItem After transmission round $k$, $\bar{\eta}$ can be updated as:
		\item[]
		\item[] {\small $\bar{\eta} = \alpha \frac{1}{|V|} \sum_{i=1}^{|V|} \eta^k_i + (1-\alpha)\bar{\eta}$}
	\end{itemize}
\end{itemize}
%
\column{0.45\textwidth}
\begin{figure}
\centering
%\begin{textblock*}{1cm}(1cm,6cm) % {block width} (coords)
\includegraphics[width=0.75\textwidth]{Figures/overallArchitecture.pdf}
%\end{textblock*}
\end{figure}
\end{columns}
%
\end{frame}