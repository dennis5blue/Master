\begin{frame}{Problem Decomposition}
\begin{itemize}
	\myItem In Problem~\eqref{eq::formulation}, the determination of $\mathbf{X}$ is highly related to $\mathbf{Z}$, therefore, we decouple it into two sub-problems:
	\begin{itemize}
		\mySubItem I-frame selection sub-problem $\xLongrightarrow{\text{solve   }}$ $\mathbf{X}$
		\mySubItem P-frame association sub-problem $\xLongrightarrow{\text{solve   }}$ $\mathbf{Z}$
		\item[]
		\item[]
	\end{itemize}
	\begin{figure}
		\centering
		\includegraphics[width=0.6\textwidth]{Figures/procedureOfSubproblems.pdf}
	\end{figure}
\end{itemize}
\end{frame}
%%
\begin{frame}{I-Frame Selection Sub-Problem (1/3)}
\begin{itemize}
	\myItem Remove the scheduling constraints (i.e. constraints about $\mathbf{Z}$) from the original formulation
	{\small \begin{align}
		&\underset{\mathbf{X}}{\min} \sum_{i=1}^{|V|} \sum_{j \in U_i}  x_{ij} h_{ij}, & \nonumber \\
		\text{subject to} & & \nonumber \\
		&U_i = \{ j \in V | d_{ij} \leq \rho d_{j0} \}, &\forall i \in V, \nonumber \\
		&\sum_{j \in U_i} x_{ij} = 1, &\forall i \in V, \nonumber \\
		&x_{ij} \leq x_{jj}, &\forall i \in V, \forall j \in W_i, \nonumber \\
		&x_{ij} = 0, &\forall i \in V, \forall j \in V \setminus U_i, \nonumber \\
		&x_{ij} = \{0,1\}, &\forall i,j \in V.
		\label{eq::formulationSimplified}
	\end{align} }% end small
\end{itemize}
\end{frame}
%%
\begin{frame}{I-Frame Selection Sub-Problem (2/3)}
%
\only<1->{
\begin{itemize}
	\myItem The idea of constraints in Problem~\eqref{eq::formulationSimplified} is to ensure that
	\begin{itemize}
		\mySubItem Only I-frame cameras can be referenced
		\mySubItem Each P-frame camera must reference from one overheard I-frame camera
	\end{itemize}
\end{itemize}
}
\only<3>{
\begin{itemize}
	\myItem Define $u_{ij}$ such that $u_{ij} = 1$ if $j \in U_i$, otherwise, $u_{ij} = 0$
	{\small \begin{align}
		&\underset{diag(\mathbf{X})}{\min} \sum_{i=1}^{|V|} x_{ii}h_{ii} + \sum_{i=1}^{|V|} (1-x_{ii}) \hat{h}_i, & \nonumber \\
		\text{subject to} & & \nonumber \\
		&\sum_{j=1}^{|V|} u_{ij} x_{jj} \geq 1, &\forall i \in V, \nonumber \\
		&x_{jj} = \{0,1\}, &\forall j \in V  
		\label{eq::IframeSelectionSubProblem}
	\end{align} }% end small
\end{itemize}
}
%
\only<2->{
\begin{textblock*}{0.9\textwidth}(2cm,3.1cm)
\begin{beamercolorbox}[center,shadow=false,rounded=true,ht=2.65em]{myNoteColor}
\begin{itemize}
	\mySubItem Each P-frame camera must have at least one I-frame camera that it can overhear
\end{itemize}
\end{beamercolorbox}
\end{textblock*}
}
%
\only<3>{
\begin{textblock*}{2.6cm}(8.25cm,5.3cm)
\begin{figure}
\centering
\includegraphics[width=0.4\textwidth]{Figures/arrow.pdf}
\end{figure}
\end{textblock*}
%
\begin{textblock*}{2.6cm}(9.85cm,6cm)
\begin{beamercolorbox}[center,shadow=true,rounded=true]{myNoteColor}
{\small $\hat{h}_i = \underset{u_{ij}x_{jj} = 1}{\min} h_{ij}$ }
\end{beamercolorbox}
\end{textblock*}
}
%
\end{frame}
%%
\begin{frame}{I-Frame Selection Sub-Problem (3/3)}
\begin{itemize}
	\myItem I-frame selection sub-problem belongs to a binary integer programming problem
	\begin{itemize}
		\mySubItem Optimal solution can be found if we search through the whole solution space
		\mySubItem We thus propose a branch-and-bound selection algorithm
		\mySubItem We also propose a heuristic approach based on graph approximation for getting a near optimal solution
	\end{itemize}
\end{itemize}
\end{frame}