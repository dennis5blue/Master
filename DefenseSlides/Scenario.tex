\begin{frame}{Network Scenario (1/3)}
\begin{itemize}
	\myItem In order to exploit spatial reuse, we proposed a hierarchical \emph{two-hop relay} communication model for uplink transmission in our previous work~\footfullcite{VTCCluster}
	\myItem However, the data correlation ``within one cluster'' has not yet been investigated
	\myItem Therefore, we try to leverage \\
			the correlation between \\
			cameras for bandwidth- \\
			efficient data gathering\\
			``within one cluster''
\end{itemize}
%
\begin{figure}
\centering
\begin{textblock*}{1cm}(6.5cm,4.9cm) % {block width} (coords)
\includegraphics[width=5.5cm]{Figures/model.pdf}
\end{textblock*}
\end{figure}
\end{frame}
%%
\begin{frame}{Network Scenario (2/3)}
\begin{itemize}
	\myItem We consider the scenario that a data aggregator is in charge of gathering video frames from cameras through direct wireless communication links
	\myItem We assume that each camera is allocated one dedicated time slot for transmission
	\begin{itemize}
		\mySubItem Each camera has the \\
				   possibility to overhear \\
				   previous transmission \\
				   for the sake of reducing \\
				   the required packet size \\
				   for encoding its video \\
				   frame
	\end{itemize}
\end{itemize}
%
\begin{figure}
\centering
\begin{textblock*}{1cm}(6.3cm,5.1cm) % {block width} (coords)
\includegraphics[width=6cm]{Figures/topo.pdf}
\end{textblock*}
\end{figure}
%
\end{frame}
%%
\begin{frame}{Network Scenario (3/3)}
\begin{columns}
	\column{0.7\textwidth}
	\begin{itemize}
		\myItem A video frame can be overheard if
		\begin{itemize}
			\mySubItem This frame is transmitted in advance
			\mySubItem This frame can be successfully received
		\end{itemize}
		\myItem After receiving one overheard video frame, we can
		\begin{itemize}
			\mySubItem Reconstruct the overheard frame if this frame is an I-frame
			\mySubItem Use MVC for inter-view prediction if the overheard frame can be reconstructed
		\end{itemize}
	\end{itemize}
	\begin{beamercolorbox}[center,shadow=true,rounded=true]{myEmphColor}
	Our problem thus includes the determination of (a) cameras schedule, and \\ (b) set of I-frame cameras
	\end{beamercolorbox}
 
	\column{0.3\textwidth}
	\begin{figure}
	\centering
	\includegraphics[width=0.9\textwidth]{Figures/scenarioIntro.pdf}
	\end{figure}
\end{columns}
\end{frame}