%% Example for using box
\begin{frame}{Box}
\begin{beamerboxesrounded}[scheme=alera,shadow=true]{Theorem}
Timmy is PHD
\end{beamerboxesrounded}
\end{frame}

%% Example for animation
\begin{frame}{Animation}
\begin{itemize}
\item If timmy \pause
\item study for PHD \pause
\item I will \pause
\item be happy
\end{itemize}
\end{frame}

%% Example for animation and color itemize
\begin{frame}{Color animation}
\begin{itemize}
\hilite<3> \item If timmy \pause
\hilite<4> \item study for PHD \pause
\hilite<5> \item I will \pause
\hilite<6> \item be happy
\end{itemize}
\end{frame}

%% Example for font size
\begin{frame}{Font size}
Hi \\
\tiny{Hi} \\
\large{Hi} \\
\Large{Hi} \\
\huge{Hi} \\
\Huge{Hi} 
\end{frame}

%% Example of multiple columns
\begin{frame}
\frametitle{Two-column slide}
 
\begin{columns}
 
\column{0.5\textwidth}
This is a text in first column.
$$E=mc^2$$
\begin{itemize}
\item First item
\item Second item
\end{itemize}
 
\column{0.5\textwidth}
This text will be in the second column
and on a second tought this is a nice looking
layout in some cases.
\end{columns}
\end{frame}
