%% Please use xelatex to compile
\documentclass[slidestop,compress,mathserif]{beamer}

%% Set the themes of my slides
% more themes can be found at: http://deic.uab.es/~iblanes/beamer_gallery/index.html
\usetheme{Montpellier}
\usecolortheme{whale}
\usefonttheme{professionalfonts}

%% Include useful packages
\usepackage[utf8]    {inputenc}
\usepackage[T1]      {fontenc}
\usepackage[english] {babel}
\usepackage{amsmath,amsfonts,graphicx,mathrsfs,wrapfig}
\usepackage{tgbonum} % for changing fonts
\usepackage{amssymb} % for using ding{} symbols
\usepackage{pifont} % for using ding{} symbols
\usepackage[absolute,overlay]{textpos} % for textblocks

%% Define my own item
% more info about ding symbols: http://tex.stackexchange.com/questions/42619/x-mark-to-match-checkmark
\newcommand*{\myItem}{\item[\ding{98}]}
\newcommand*{\mySubItem}{\item[\ding{243}]}

%% Define my own color
\definecolor{myBlue}{RGB}{28,107,160}
\definecolor{myBlack}{RGB}{10,50,40}
\definecolor{myBrown}{RGB}{97,48,48}

%% Set the color of my slides
 % change the color for footline
 \setbeamertemplate{footline}[frame number]{}
 \setbeamercolor{footline}{fg=myBlack}
 % change the color of background
 \beamertemplateshadingbackground{white!100}{myBlue!5}

\beamertemplateballitem % set itemize type
\beamertemplatenavigationsymbolsempty % remove navigation symbols

%% Set tonic logo
\logo{\includegraphics[width=0.22\columnwidth]{Figures/NTU}\hspace{25pt}\vspace{-18pt}}

%% Define for animiating itemize
\def\hilite<#1>{\temporal<#1>{\color{gray}}{\color{blue}}{\color{blue!25}}}

%% Create my own frame title style
\setbeamercolor{myTitleColor}{fg=myBrown,bg=myBlue!35}
\setbeamerfont{frametitle}{size=\large,series=\bfseries}
\setbeamertemplate{frametitle}
{
    \nointerlineskip
    \begin{beamercolorbox}[rounded=true,sep=0.3cm,ht=1.8em,wd=\paperwidth]{myTitleColor}
        \vbox{}\vskip-2ex%
        \strut\insertframetitle\strut
        \vskip-0.8ex%
    \end{beamercolorbox}
}

%% Start title
\title[Exploiting Inter-View Correlation for Bandwidth-Efficient Data Gathering in Wireless Multi-Camera Networks]{Exploiting Inter-View Correlation for Bandwidth-Efficient Data Gathering in Wireless Multi-Camera Networks}
\author[]{Speaker: Chang-Yu Song \\
          Advisor: Prof. Hung-Yun Hsieh}
\date[]{July 28th, 2015}
\institute{TONIC Research Group \\
  		   Graduate Institute of Communication Engineering \\
		   National Taiwan University}
   
%% Add outline at the beginning of each sections
\AtBeginSection[]
{
  \begin{frame}
  \frametitle{Outline}
  \tableofcontents[currentsection]
  \end{frame}
}

%% Start document
\begin{document}
\maketitle

\section*{Outline}
\begin{frame}{Outline}
\begin{itemize}
\item Introduction and Related Work
\item Network Scenario and Problem Formulation
\item Correlated Image Gathering under Overhearing Source Coding
\item Correlated Video Gathering for Wireless Multi-Camera Networks
\item Conclusions and Future Work
\end{itemize}
\end{frame}

\section{Introduction and Motivation}
\addtocontents{toc}{\protect\setcounter{tocdepth}{1}}
\subsection{Introduction and Motivation}
\begin{frame}{Introduction}
\begin{itemize}
	\myItem Multi-camera networks
	\begin{itemize}
		\mySubItem fuck
		\mySubItem gg
		\mySubItem FUCK KMT
	\end{itemize}
\end{itemize}
\begin{figure}
\centering
\includegraphics[width=0.8\textwidth]{Figures/multiCamNetwork.pdf}
\end{figure}
\end{frame}

\begin{frame}{Motivation}
%
\begin{itemize}
	\myItem Limitation for available radio resource
	\begin{itemize}
		\mySubItem Data quantity for multimedia applications are often much more larger than conventional scalar sensor networks
		\mySubItem A large amount of cameras have to share the radio resource with human devices
	\end{itemize}
	\myItem Inter-view correlation
	\begin{itemize}
		\mySubItem Nearby cameras observing the same area might produce correlated video streams
		\mySubItem Multiview video coding (MVC) technique can be exploited to remove \emph{inter-view correlation}
	\end{itemize}
	\myItem Due to the resource limitation and video correlation, we should focus on serving the ``data'' instead of individual cameras
\end{itemize}
%
	\begin{wrapfigure}{l}{0.3\textwidth}
		\centering
		\includegraphics[width=0.3\textwidth]{Figures/multiCam.pdf}
		%\rule{0.9\linewidth}{0.75\linewidth}
	\end{wrapfigure}
%
\end{frame}
\begin{frame}{Related Work}
%
\begin{itemize}
	\myItem Multi-camera networks
	\begin{itemize}
		\mySubItem C. Yeo et al proposed an idea about exploiting inter-view correlation to provide robust video delivery from distributed coded wireless cameras~\footfullcite{RobustMVCinWCN}
		\mySubItem GG~\footfullcite{CorrAwareScheduling}
	\end{itemize}
\end{itemize}
%
\end{frame}
%%
\begin{frame}{Related Work}
\begin{itemize}
	\myItem Overhearing source coding
	\begin{itemize}
		\mySubItem GG~\footfullcite{BodyAreaSensorNetwork}
		\mySubItem GG~\footfullcite{imageModelCluster}
	\end{itemize}
\end{itemize}
\end{frame}
\subsection{Related Work}


\section{Network Scenario and Camera Model}
\addtocontents{toc}{\protect\setcounter{tocdepth}{2}}

\section{Correlated Data Gathering under Overhearing Source Coding}
\addtocontents{toc}{\protect\setcounter{tocdepth}{3}}

\section{Correlated Data Gathering for Multi-Camera Networks}
\addtocontents{toc}{\protect\setcounter{tocdepth}{4}}

\section{Conclusions and Future Work}
\addtocontents{toc}{\protect\setcounter{tocdepth}{5}}

%% Example for using box
\begin{frame}{Box}
\begin{beamerboxesrounded}[scheme=alera,shadow=true]{Theorem}
Timmy is PHD
\end{beamerboxesrounded}
\end{frame}

%% Example for animation
\begin{frame}{Animation}
\begin{itemize}
\item If timmy \pause
\item study for PHD \pause
\item I will \pause
\item be happy
\end{itemize}
\end{frame}

%% Example for animation and color itemize
\begin{frame}{Color animation}
\begin{itemize}
\hilite<3> \item If timmy \pause
\hilite<4> \item study for PHD \pause
\hilite<5> \item I will \pause
\hilite<6> \item be happy
\end{itemize}
\end{frame}

%% Example for font size
\begin{frame}{Font size}
Hi \\
\tiny{Hi} \\
\large{Hi} \\
\Large{Hi} \\
\huge{Hi} \\
\Huge{Hi} 
\end{frame}

%% Example of multiple columns
\begin{frame}
\frametitle{Two-column slide}
 
\begin{columns}
 
\column{0.5\textwidth}
This is a text in first column.
$$E=mc^2$$
\begin{itemize}
\item First item
\item Second item
\end{itemize}
 
\column{0.5\textwidth}
This text will be in the second column
and on a second tought this is a nice looking
layout in some cases.
\end{columns}
\end{frame}

\end{document}

