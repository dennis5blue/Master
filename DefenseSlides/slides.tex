%% Please use xelatex to compile
\documentclass[slidestop,compress,mathserif]{beamer}

%% Set the themes of my slides
% more themes can be found at: http://deic.uab.es/~iblanes/beamer_gallery/index.html
\usetheme{Montpellier}
\usecolortheme{whale}
\usefonttheme{professionalfonts}

%% Include useful packages
\usepackage[utf8]    {inputenc}
\usepackage[T1]      {fontenc}
\usepackage[english] {babel}
\usepackage{amsmath,amsfonts,graphicx,mathrsfs,wrapfig}
\usepackage[backend=bibtex,style=ieee,doi=false]{biblatex}
\bibliography{reference}
\usepackage{tgbonum} % for changing fonts
\usepackage{amssymb} % for using ding{} symbols
\usepackage{pifont} % for using ding{} symbols
\usepackage[absolute,overlay]{textpos} % for textblocks
\usepackage{extarrows} % for fancy arrows
\usepackage{tcolorbox} % for tcolorbox

%% Define my own item
% more info about ding symbols: http://tex.stackexchange.com/questions/42619/x-mark-to-match-checkmark
\newcommand*{\myItem}{\item[\ding{96}]} %98
\newcommand*{\mySubItem}{\item[\ding{226}]} %243

%% Define my own color
\definecolor{myBlue}{RGB}{28,107,160}
\definecolor{myBlack}{RGB}{10,50,40}
\definecolor{myBrown}{RGB}{97,48,48}
\definecolor{myGray}{RGB}{123,123,123}
\definecolor{myOrange}{RGB}{217,70,0}
\definecolor{myRed}{RGB}{206,0,0}
\setbeamercolor{myEmphColor}{fg=myOrange,bg=myGray!20}
\setbeamercolor{myNoteColor}{fg=myRed,bg=myGray!10}

%% Set the color of my slides
 % change the color for footline
 \setbeamertemplate{footline}[frame number]{}
 \setbeamercolor{footline}{fg=myBlack}
 % change the color of background
 \beamertemplateshadingbackground{white!100}{myBlue!5}

\beamertemplateballitem % set itemize type
\beamertemplatenavigationsymbolsempty % remove navigation symbols

%% Set tonic logo
\logo{\includegraphics[width=0.22\columnwidth]{Figures/NTU}\hspace{25pt}\vspace{-18pt}}

%% Define for animiating itemize
\def\hilite<#1>{\temporal<#1>{\color{gray}}{\color{blue}}{\color{blue!25}}}

%% Create my own frame title style
\setbeamercolor{myTitleColor}{fg=myBrown,bg=myBlue!35}
\setbeamerfont{frametitle}{size=\large,series=\bfseries}
\setbeamertemplate{frametitle}
{
    \nointerlineskip
    \begin{beamercolorbox}[rounded=true,sep=0.3cm,ht=1.8em,wd=\paperwidth]{myTitleColor}
        \vbox{}\vskip-2ex%
        \strut\bf{\insertframetitle}\strut
        \vskip-0.8ex%
    \end{beamercolorbox}
}

%% Change font size for foot citation
\renewcommand{\footnotesize}{\tiny}

%% Start title
\title[Exploiting Inter-View Correlation for Bandwidth-Efficient Data Gathering in Wireless Multi-Camera Networks]{Exploiting Inter-View Correlation for Bandwidth-Efficient Data Gathering in Wireless Multi-Camera Networks}
\author[]{Speaker: Chang-Yu Song \\
          Advisor: Prof. Hung-Yun Hsieh}
\date[]{July 28th, 2015}
\institute{TONIC Research Group \\
  		   Graduate Institute of Communication Engineering \\
		   National Taiwan University}
   
%% Add outline at the beginning of each sections
\AtBeginSection[]
{
  \begin{frame}
  \frametitle{Outline}
  \tableofcontents[currentsection]
  \end{frame}
}

%% Start document
\begin{document}
\maketitle

\section*{Outline}
\begin{frame}{Outline}
\begin{itemize}
\item Introduction and Related Work
\item Network Scenario and Problem Formulation
\item Correlated Image Gathering under Overhearing Source Coding
\item Correlated Video Gathering for Wireless Multi-Camera Networks
\item Conclusions and Future Work
\end{itemize}
\end{frame}

\section{Introduction and Related Work}
\addtocontents{toc}{\protect\setcounter{tocdepth}{1}}
%
\subsection{Introduction and Motivation}
\begin{frame}{Introduction}
\begin{itemize}
	\myItem Multi-camera networks
	\begin{itemize}
		\mySubItem fuck
		\mySubItem gg
		\mySubItem FUCK KMT
	\end{itemize}
\end{itemize}
\begin{figure}
\centering
\includegraphics[width=0.8\textwidth]{Figures/multiCamNetwork.pdf}
\end{figure}
\end{frame}

\begin{frame}{Motivation}
%
\begin{itemize}
	\myItem Limitation for available radio resource
	\begin{itemize}
		\mySubItem Data quantity for multimedia applications are often much more larger than conventional scalar sensor networks
		\mySubItem A large amount of cameras have to share the radio resource with human devices
	\end{itemize}
	\myItem Inter-view correlation
	\begin{itemize}
		\mySubItem Nearby cameras observing the same area might produce correlated video streams
		\mySubItem Multiview video coding (MVC) technique can be exploited to remove \emph{inter-view correlation}
	\end{itemize}
	\myItem Due to the resource limitation and video correlation, we should focus on serving the ``data'' instead of individual cameras
\end{itemize}
%
	\begin{wrapfigure}{l}{0.3\textwidth}
		\centering
		\includegraphics[width=0.3\textwidth]{Figures/multiCam.pdf}
		%\rule{0.9\linewidth}{0.75\linewidth}
	\end{wrapfigure}
%
\end{frame}
\subsection{Related Work}
\begin{frame}{Related Work}
%
\begin{itemize}
	\myItem Multi-camera networks
	\begin{itemize}
		\mySubItem C. Yeo et al proposed an idea about exploiting inter-view correlation to provide robust video delivery from distributed coded wireless cameras~\footfullcite{RobustMVCinWCN}
		\mySubItem GG~\footfullcite{CorrAwareScheduling}
	\end{itemize}
\end{itemize}
%
\end{frame}
%%
\begin{frame}{Related Work}
\begin{itemize}
	\myItem Overhearing source coding
	\begin{itemize}
		\mySubItem GG~\footfullcite{BodyAreaSensorNetwork}
		\mySubItem GG~\footfullcite{imageModelCluster}
	\end{itemize}
\end{itemize}
\end{frame}

\section{Network Scenario and Problem Formulation}
\addtocontents{toc}{\protect\setcounter{tocdepth}{1}}
%
\subsection{Network Scenario}
\begin{frame}{Network Scenario (1/3)}
\begin{itemize}
	\myItem In order to exploit spatial reuse, we proposed a hierarchical \emph{two-hop relay} communication model for uplink transmission in our previous work~\footfullcite{VTCCluster}
	\myItem However, the data correlation ``within one cluster'' has not yet been investigated
	\myItem Therefore, we try to leverage \\
			the correlation between \\
			cameras for bandwidth- \\
			efficient data gathering\\
			``within one cluster''
\end{itemize}
%
\begin{figure}
\centering
\begin{textblock*}{1cm}(6.5cm,4.9cm) % {block width} (coords)
\includegraphics[width=5.5cm]{Figures/model.pdf}
\end{textblock*}
\end{figure}
\end{frame}
%%
\begin{frame}{Network Scenario (2/3)}
\begin{itemize}
	\myItem We consider the scenario that a data aggregator is in charge of gathering video frames from cameras through direct wireless communication links
	\myItem We assume that each camera is allocated one dedicated time slot for transmission
	\begin{itemize}
		\mySubItem Each camera has the \\
				   possibility to overhear \\
				   previous transmission \\
				   for the sake of reducing \\
				   the required packet size \\
				   for encoding its video \\
				   frame
	\end{itemize}
\end{itemize}
%
\begin{figure}
\centering
\begin{textblock*}{1cm}(6.3cm,5.1cm) % {block width} (coords)
\includegraphics[width=6cm]{Figures/topo.pdf}
\end{textblock*}
\end{figure}
%
\end{frame}
%%
\begin{frame}{Network Scenario (3/3)}
\begin{columns}
	\column{0.7\textwidth}
	\begin{itemize}
		\myItem A video frame can be overheard if
		\begin{itemize}
			\mySubItem This frame is transmitted in advance
			\mySubItem This frame can be successfully received
		\end{itemize}
		\myItem After receiving one overheard video frame, we can
		\begin{itemize}
			\mySubItem Reconstruct the overheard frame if this frame is an I-frame
			\mySubItem Use MVC for inter-view prediction if the overheard frame can be reconstructed
		\end{itemize}
	\end{itemize}
	\begin{beamercolorbox}[center,shadow=true,rounded=true]{myEmphColor}
	Our problem thus includes the determination of (a) cameras schedule, and \\ (b) set of I-frame cameras
	\end{beamercolorbox}
 
	\column{0.3\textwidth}
	\begin{figure}
	\centering
	\includegraphics[width=0.9\textwidth]{Figures/scenarioIntro.pdf}
	\end{figure}
\end{columns}
\end{frame}
\subsection{Problem Formulation}
\begin{frame}{Problem Formulation (1/4)}
\begin{itemize}
	\myItem Consider $V=\{1,2,\cdots,|V|\}$ as the set of cameras deployed for gathering video data, where camera $i$ will produce video frame $F_i$
	\begin{itemize}
		\mySubItem $F_i$ can be encoded as either I-frame or P-frame
		\mySubItem $h_{ii}$ and $h_{ij}$ denotes the amount of bits required to encode $F_i$ as an I-frame or a P-frame referenced from $F_j$, respectively
	\end{itemize}	
	\myItem Denote $\mathbf{X}=[x_{ij}]_{|V| \times |V|}$ as decision variables where $x_{ij} = 1$ indicates $F_i$ will reference from $F_j$, otherwise, $x_{ij} = 0$
	\begin{itemize}
		\mySubItem Our goal can be written as:
		\begin{equation}
			\underset{\mathbf{X}}{\min} \sum_{i=1}^{|V|} \sum_{j =1}^{|V|}  x_{ij} h_{ij}
		\end{equation}
		\mySubItem However, some constraints are necessary to be considered
	\end{itemize}
\end{itemize}
\end{frame}
%%
\begin{frame}{Problem Formulation (2/4)}
\begin{itemize}
	\myItem The set of cameras that camera $i$ ``has the possibility'' to overhear is
	\begin{equation}
		U_i = \{ j \in V | \tau C_{ji} \geq h_{jj} \} \quad \Rightarrow \quad U_i = \{ j \in V | d_{ij} \leq \rho d_{j0} \}
		\label{eq::inTransRange}
	\end{equation}
	\myItem Denote ${\mathbf{Z} = [z_1, z_2, \cdots, z_{|V|}]}$, ${1 \leq z_i \leq N}$, ${i=1,\cdots,|V|}$ as the scheduling vector such that ${z_i = n}$ if camera $i$ is scheduled at slot $n$.
	\myItem According to $\mathbf{Z}$ and \eqref{eq::inTransRange}, The set of cameras that camera $i$ ``can'' overhear can be written as:
	\begin{equation}
		W_i = \{ j \in U_i | z_j \leq z_i \}
		\label{eq::prevSet}
	\end{equation}
\end{itemize}
\end{frame}
%%
\begin{frame}{Problem Formulation (3/4)}
\begin{itemize}
	\myItem All cameras has one dedicated transmission slot
	\begin{equation}
		|z_i - z_j| > 0, \forall i,j \in V, i \neq j
		\label{eq::oneSlotOneCam}
	\end{equation}
	\myItem All cameras must transmit its video frame
	\begin{equation}
		\sum_{j \in W_i} x_{ij} = 1, \forall i \in V
		\label{eq::referenceConstraint}
	\end{equation}
	\myItem Only overheard I-frame can be referenced
	\begin{equation} \left\{ \begin{array}{ll}
		x_{ij} = 0, &\forall i \in V, \forall j \in V \setminus W_i \\
		x_{ij} \leq x_{jj}, &\forall i \in V, \forall j \in W_i \end{array} \right.
		\label{eq::referenceOnlyIframe}
	\end{equation}
\end{itemize}
\end{frame}
%%
\begin{frame}{Problem Formulation (4/4)}
\begin{itemize}
	\myItem The overall formulation can now be written as:
	{\small	\begin{align}
		&\underset{\mathbf{X},\mathbf{Z}}{\min} \sum_{i=1}^{|V|} \sum_{j \in W_i}  x_{ij} h_{ij}, & \nonumber \\
\text{subject to} & & \nonumber \\
		&U_i = \{ j \in V | d_{ij} \leq \rho d_{j0} \}, &\forall i \in V, \nonumber \\
		&W_i = \{ j \in U_i | z_j \leq z_i \}, &\forall i \in V, \nonumber \\
		&\sum_{j \in W_i} x_{ij} = 1, &\forall i \in V, \nonumber \\
		&x_{ij} \leq x_{jj}, &\forall i \in V, \forall j \in W_i, \nonumber \\
		&x_{ij} = 0, &\forall i \in V, \forall j \in V \setminus W_i, \nonumber \\
		&|z_i - z_j| > 0, &\forall i,j \in V, i \neq j
\label{eq::formulation}
	\end{align} } % end small
	\myItem $\mathbf{X}$ and $\mathbf{Z}$ must belong to positive integers
\end{itemize}
\end{frame}

\section{Correlated Data Gathering under Overhearing Source Coding}
\addtocontents{toc}{\protect\setcounter{tocdepth}{1}}
%
\subsection{Problem Decomposition}
\begin{frame}{Problem Decomposition}
\begin{itemize}
	\myItem In Problem~\eqref{eq::formulation}, the determination of $\mathbf{X}$ is highly related to $\mathbf{Z}$, therefore, we decouple it into two sub-problems:
	\begin{itemize}
		\mySubItem I-frame selection sub-problem $\xLongrightarrow{\text{solve   }}$ $\mathbf{X}$
		\mySubItem P-frame association sub-problem $\xLongrightarrow{\text{solve   }}$ $\mathbf{Z}$
		\item[]
		\item[]
	\end{itemize}
	\begin{figure}
		\centering
		\includegraphics[width=0.6\textwidth]{Figures/procedureOfSubproblems.pdf}
	\end{figure}
\end{itemize}
\end{frame}
%%
\begin{frame}{I-Frame Selection Sub-Problem (1/3)}
\begin{itemize}
	\myItem Remove the scheduling constraints (i.e. constraints about $\mathbf{Z}$) from the original formulation
	{\small \begin{align}
		&\underset{\mathbf{X}}{\min} \sum_{i=1}^{|V|} \sum_{j \in U_i}  x_{ij} h_{ij}, & \nonumber \\
		\text{subject to} & & \nonumber \\
		&U_i = \{ j \in V | d_{ij} \leq \rho d_{j0} \}, &\forall i \in V, \nonumber \\
		&\sum_{j \in U_i} x_{ij} = 1, &\forall i \in V, \nonumber \\
		&x_{ij} \leq x_{jj}, &\forall i \in V, \forall j \in W_i, \nonumber \\
		&x_{ij} = 0, &\forall i \in V, \forall j \in V \setminus U_i, \nonumber \\
		&x_{ij} = \{0,1\}, &\forall i,j \in V.
		\label{eq::formulationSimplified}
	\end{align} }% end small
\end{itemize}
\end{frame}
%%
\begin{frame}{I-Frame Selection Sub-Problem (2/3)}
%
\only<1->{
\begin{itemize}
	\myItem The idea of constraints in Problem~\eqref{eq::formulationSimplified} is to ensure that
	\begin{itemize}
		\mySubItem Only I-frame cameras can be referenced
		\mySubItem Each P-frame camera must reference from one overheard I-frame camera
	\end{itemize}
\end{itemize}
}
\only<3>{
\begin{itemize}
	\myItem Define $u_{ij}$ such that $u_{ij} = 1$ if $j \in U_i$, otherwise, $u_{ij} = 0$
	{\small \begin{align}
		&\underset{diag(\mathbf{X})}{\min} \sum_{i=1}^{|V|} x_{ii}h_{ii} + \sum_{i=1}^{|V|} (1-x_{ii}) \hat{h}_i, & \nonumber \\
		\text{subject to} & & \nonumber \\
		&\sum_{j=1}^{|V|} u_{ij} x_{jj} \geq 1, &\forall i \in V, \nonumber \\
		&x_{jj} = \{0,1\}, &\forall j \in V  
		\label{eq::IframeSelectionSubProblem}
	\end{align} }% end small
\end{itemize}
}
%
\only<2->{
\begin{textblock*}{0.9\textwidth}(2cm,2.9cm)
\begin{beamercolorbox}[center,shadow=false,rounded=true,ht=2.65em]{myNoteColor}
\begin{itemize}
	\mySubItem Each P-frame camera must have at least one I-frame camera that it can overhear
\end{itemize}
\end{beamercolorbox}
\end{textblock*}
}
%
\only<3>{
\begin{textblock*}{2.6cm}(8.25cm,5.3cm)
\begin{figure}
\centering
\includegraphics[width=0.4\textwidth]{Figures/arrow.pdf}
\end{figure}
\end{textblock*}
%
\begin{textblock*}{2.6cm}(9.85cm,6cm)
\begin{beamercolorbox}[center,shadow=true,rounded=true]{myNoteColor}
{\small $\hat{h}_i = \underset{u_{ij}x_{jj} = 1}{\min} h_{ij}$ }
\end{beamercolorbox}
\end{textblock*}
}
%
\end{frame}
%%
\begin{frame}{I-Frame Selection Sub-Problem (3/3)}
\begin{itemize}
	\myItem I-frame selection sub-problem belongs to a binary integer programming problem
	\begin{itemize}
		\mySubItem Optimal solution can be found if we search through the whole solution space
		\mySubItem We thus propose a branch-and-bound selection algorithm
		\mySubItem We also propose a heuristic approach based on graph approximation for getting a near optimal solution
	\end{itemize}
\end{itemize}
\end{frame}
\subsection{Proposed Branch-and-Bound Selection Algorithm}
\begin{frame}{Proposed Branch-and-Bound Selection Algorithm}
\begin{itemize}
	\myItem Main idea:
	\begin{itemize}
		\mySubItem If 
	\end{itemize}
\end{itemize}
%
\begin{figure}
\centering
\begin{textblock*}{1cm}(5.5cm,4.8cm) % {block width} (coords)
\includegraphics[width=7cm]{Figures/BBdesignIframeSelection.pdf}
\end{textblock*}
\end{figure}
%
\end{frame}

\section{Correlated Data Gathering for Multi-Camera Networks}
\addtocontents{toc}{\protect\setcounter{tocdepth}{1}}

\section{Conclusions and Future Work}
\addtocontents{toc}{\protect\setcounter{tocdepth}{1}}
%% Example for using box
\begin{frame}{Box}
\begin{beamerboxesrounded}[scheme=alera,shadow=true]{Theorem}
Timmy is PHD
\end{beamerboxesrounded}
\end{frame}

%% Example for animation
\begin{frame}{Animation}
\begin{itemize}
\item If timmy \pause
\item study for PHD \pause
\item I will \pause
\item be happy
\end{itemize}
\end{frame}

%% Example for animation and color itemize
\begin{frame}{Color animation}
\begin{itemize}
\hilite<3> \item If timmy \pause
\hilite<4> \item study for PHD \pause
\hilite<5> \item I will \pause
\hilite<6> \item be happy
\end{itemize}
\end{frame}

%% Example for font size
\begin{frame}{Font size}
Hi \\
\tiny{Hi} \\
\large{Hi} \\
\Large{Hi} \\
\huge{Hi} \\
\Huge{Hi} 
\end{frame}

%% Example of multiple columns
\begin{frame}
\frametitle{Two-column slide}
 
\begin{columns}
 
\column{0.5\textwidth}
This is a text in first column.
$$E=mc^2$$
\begin{itemize}
\item First item
\item Second item
\end{itemize}
 
\column{0.5\textwidth}
This text will be in the second column
and on a second tought this is a nice looking
layout in some cases.
\end{columns}
\end{frame}

\end{document}