\section{Proposed I-frame selection Algorithm}
\label{sec::IFrameSelection}
%\begin{figure*}
\begin{figure}
\begin{center}
\includegraphics[width=0.95\columnwidth]{Figures/BBdesignIframeSelection.pdf}
\caption{\label{fig::BBdesgin}Enumeration tree of branch-and-bound algorithm}
\end{center}
\end{figure}
%\end{figure*}
%
\begin{figure}
\begin{center}
\includegraphics[width=0.95\columnwidth]{Figures/referenceStructure.pdf}
\caption{\label{fig::encodeOrder}Coding structure of a group of pictures}
\end{center}
\end{figure}
As we mentioned before, problem~\eqref{eq::formulation} is a $\mathcal{NP}$-hard problem, therefore, we show how to simplify and solve this problem in this chapter.
Note that we assume a P-frame can only reference from one I-frame as equation~\eqref{eq::referenceOnlyIframe} shows.
According to equation~\eqref{eq::referenceOnlyIframe}, the coding structure of a group of pictures can be shown in figure~\ref{fig::encodeOrder}, where every P-frames will reference from the closest previous I-frame to reduce its encoded bits.
Consequently, problem~\eqref{eq::formulation} can be decomposed into two sub-problems, including (a) the I-frame selection sub-problem and (b) the P-frame scheduling sub-problem.
We show how we solve the I-frame selection sub-problem in this chapter, where the proposed algorithm for P-frame scheduling is presented in chapter~\ref{sec::PFrameScheduling}.

\subsection{I-frame Selection Sub-Problem}
\label{sec::iFrameSelectionSubProb}
Note that the decision variables of the I-frame selection sub-problem are the diagonal elements of $\mathbf{X}$.
We aim to figure out whether ${x_{ii},i \in V}$ equals to $0$ or $1$ so that the total amount of encoded bits can be minimized, where the other scheduling constraints are ignore during the selection of I-frames.
Suppose that camera $i$ is encoded as an I-frame, it is clearly that it requires $H(F_i)$ to encode its image.
However, if camera $i$ is encoded as a P-frame, the amount of encoded bits should depend on which I-frame it references from.
An intuitive method to determine the reference frame is to select the most correlated I-frame, and the amount of encoded bits can be written as $\underset{j \in U_i, x_{jj}=1}{\min} H(F_i|F_j)$.
Therefore, the I-frame selection sub-problem can be written as
\begin{equation*}
\min \left(
\sum_{i=1}^{|V|} x_{ii}H(F_i) +
\sum_{i=1}^{|V|} (1-x_{ii}) \underset{j \in D_i}{\min} H(F_i|F_j) \right),
\end{equation*}
subject to
\begin{align}
&U_i = \{ j \in V | d_{ij} \leq \rho d_{j0} \}, &\forall i \in V, \nonumber \\
&D_i = \{ j \in U_i| x_{jj} = 1\}, &\forall i \in V, \nonumber \\
&x_{ii} = \{0,1\}, &\forall i \in V.
\label{eq::IframeSelectionSubProblem}
\end{align}

Since all the decision variables ${x_{ii},i \in V}$ belong to a binary integer, it is clearly that the optimal solution can be found if we search through the feasible set.
However, directly traversing the whole set would lead to very high computational complexity, and hence some policy must be included for reducing the complexity.
One possible method without losing the optimality is through the branch-and-bound (BB) algorithm~\cite{BB} as presented in the followings.
Another algorithm to solve problem~\eqref{eq::IframeSelectionSubProblem} in a more time efficient way is to model~\eqref{eq::IframeSelectionSubProblem} into a graph problem, which will also be described in this chapter.
%
\subsection{Branch-and-Bound Selection Algorithm}
{\color{red}Describe the proposed branch-and-bound algorithm for I-frame selection}
Note that the solution set of problem~\eqref{eq::IframeSelectionSubProblem} can be modeled as an enumeration tree as shown in figure~\ref{fig::BBdesgin}, where the root of the enumeration tree is denoted as the whole solution set $\mathcal{S}$.
The main idea of the BB algorithm is to divide the solution set $\mathcal{S}$ into smaller subsets $\mathcal{S}^1, \mathcal{S}^2, \cdots$ and if one subset has no possibility to include the optimal solution, then it is not required to be divided any more.
Therefore, a proper policy to ``branch'' and ``bound'' must be well designed for the efficiency of the BB algorithm, and we introduce the ``branching'' and ``bounding'' policy in the followings.
%
\subsubsection{Bounding Policy}
Before introducing the bounding policy, we first propose an algorithm to estimate the lower bound of total cost of each sub solution set $\mathcal{S}^t$.
If the cost lower bound of a sub solution set $lb(\mathcal{S}^t)$ is greater than the upper bound of objective value, the sub solution set $\mathcal{S}^t$ is no longer necessary to be divided since it has no possibility to contain the optimal solution.
Therefore, the time complexity of the BB algorithm is highly correlated to the difference between estimated lower bound and actual lower bound.
For this reason, a smart algorithm is thus necessary to be designed for finding a tighter lower bound.
To start, we denote the cost matrix for the I-frame selection sub-problem as:
\begin{align}
\mathbf{H}
&= \left[ \mathbf{h}_1 \quad \mathbf{h}_2 \quad \cdots \quad \mathbf{h}_{|V|} \right], \nonumber \\
&= \left[ \begin{array}{cccc}
h_{11} &h_{12} &\cdots &h_{1|V|} \\
h_{21} &h_{22} &\cdots &h_{2|V|} \\
\vdots &\vdots &\vdots &\vdots \\
h_{|V|1} &h_{|V|2} &\cdots &h_{|V||V|}
\end{array} \right],
\label{eq::bbCostMatrix}
\end{align}
where $\mathbf{h}_k$ is the $k^{th}$ column of $\mathbf{H}$ and ${h_{ij} = H(F_i|F_j)}$ if ${i \neq j}$ and ${h_{ii}=H(F_i|F_i)=H(F_i)}$.
The cost matrix in~\eqref{eq::bbCostMatrix} will be calculated as a beforehand information so that the lower bound of cost below a branch can be estimated directly through $\mathbf{H}$.

For an arbitrary sub solution set $\mathcal{S}^t$, some decision variables are fixed to either $0$ or $1$ while the others are still undetermined.
We thus denote the partial determined vector ${\tilde{\mathbf{X}^t} = \{ \tilde{x_{11}^t},\cdots, \tilde{x_{|V||V|}^t} \} }$ such that
\begin{equation*}
\left\{ \begin{array}{ll}
\tilde{x_{ii}^t} = 1,  &\text{ if camera $i$ is determined to encode as an} \\
                   	   &\text{ I-frame for $\mathcal{S}^t$ and all its sub solution set,} \\
\tilde{x_{ii}^t} = 0,  &\text{ if camera $i$ is determined to encode as a} \\
                   	   &\text{ P-frame for $\mathcal{S}^t$ and all its sub solution set,} \\
\tilde{x_{ii}^t} = -1, &\text{ if the camera $i$'s encoding style is not decided} \\
                       &\text{ until the division of sub solution set $\mathcal{S}^t$.}
\end{array} \right.
\end{equation*}
%
According to $\tilde{\mathbf{X}^t}$, we can further define the determined I-frame set of $\mathcal{S}^t$ as $\mathcal{I}^t$:
\begin{equation}
\mathcal{I}^t = \{ i \in V | \tilde{x_{ii}^t} = 1 \},
\label{eq::IframeSet}
\end{equation}
and the determined P-frame set can also be written as:
\begin{equation}
\mathcal{P}^t = \{ i \in V | \tilde{x_{ii}^t} = 0 \},
\label{eq::IframeSet}
\end{equation}
where the set of undetermined encoded style cameras at $\mathcal{S}^t$ is $V \setminus \{ \mathcal{I}^t \cup \mathcal{P}^t \}$.

For camera $i \in \mathcal{I}^t$, it is clearly that the amount of bits required to encoded $F_i$ equals $H(F_i)$.
On the other hand, if $k \in V \setminus \mathcal{I}^t$, camera $k$ has the possibility to encode its image as a P-frame.
As a consequence, the amount of bits required to encode $F_k$ must be less than or equal to $H(F_k)$.
Since our goal is to find the \emph{lower bound} of required encoded bits, we can thus relaxed the constraints of P-frame reference from an I-frame so that the obtained result will become a weak lower bound.
That is, the cost of P-frame $\underset{j \in D_k}{\min} H(F_k|F_j)$ now becomes $\underset{j \in U_k}{\min} H(F_k|F_j)$.
However, we also note that according to constraint~\eqref{eq::referenceOnlyIframe}, a P-frame cannot become a reference frame of camera $k$, which means that the lower bound $\underset{j \in U_k}{\min} H(F_k|F_j)$ is still too optimistic.
Therefore, a tighter lower bound of camera $k$'s encoding cost should be $\underset{j \in U_k, j \notin \mathcal{P}^t}{\min} H(F_k|F_j)$.
%

Besides, also note that if camera $i$ reference from camera $j$, then camera $j$ cannot reference from camera $i$.
As a consequence, either camera $i$ or $j$ should change its reference camera so that the solution can be feasible.
Since our goal is to find the lower bound of encoding cost, which one to change the reference camera depends on the incremental cost for changing the reference camera, where the incremental cost of camera $i$ and $j$ can be written as:
\begin{align}
\Delta_i &= \left( \underset{k \in U_i, k \notin \{\mathcal{P}^t \cup j\}}{\min} H(F_i|F_k) \right) - H(F_i|F_j), \nonumber \\
\Delta_j &= \left( \underset{k \in U_j, k \notin \{\mathcal{P}^t \cup i\}}{\min} H(F_j|F_k) \right) - H(F_j|F_i).
\end{align}
Based on the incremental cost $\Delta_i$ and $\Delta_j$, we can define a selection function for determining which camera should change its reference as:
\begin{equation}
Q(i,j) = 
\left\{ \begin{array}{ll}
i,  &\text{ if $\Delta_i \leq \Delta_j$,} \\                  	   
j,  &\text{ if $\Delta_i > \Delta_j$,} \\
\end{array} \right.
\end{equation}
where $Q(i,j) = i$ means that the reference camera of camera $i$ is changed from camera $j$ to camera $\underset{k \in U_i, k \notin \{\mathcal{P}^t \cup j\}}{\arg \min} H(F_i|F_k)$ due to the reason that changing camera $i$'s reference camera has lower cost.

In short, the lower bound estimation procedure of an arbitrary branch $t$ is summarized in algorithm~\ref{alg::lbEstimation}.
%
\IncMargin{1em}
\begin{algorithm}[]
 \SetAlgoLined
 \SetKwInOut{Input}{Input}\SetKwInOut{Output}{Output}
 \Input{Cameras set $V$ and partial determined I-frame set $\mathcal{I}^t$~\eqref{eq::IframeSet} and cost matrix $\mathbf{H}$~\eqref{eq::bbCostMatrix}}
 \Output{Lower bound $lb^t$ of total amount of encoded bits below branch $t$}
 \BlankLine
 Initialize $lb^t \gets 0$\\
 \For{$i \in \mathcal{I}^t$}
 {
 	$lb^t \gets lb^t + H(F_i)$ \\
 }
 \For{$k \in V \setminus \mathcal{I}^t$}
 {
 	$lb^t \gets lb^t + \underset{j \in V}{\min} H(F_k|F_j)$ \\
 }
 \caption{\label{alg::lbEstimation}Lower bound estimation}
\end{algorithm}
\DecMargin{1em}
%
\subsubsection{Branching Policy}
The reason why BB algorithm can be applied to solve discrete and combination optimization problems is that the candidate solutions of these problems can be thought of as a rooted tree where the full solutions set is the root of this tree.
The branching phase is thus designed for separating the solution set into several subsets.
For example, we divide the solution set by fixing some elements of ${x_{ii},i \in V}$.
That is, the left hand side of figure~\ref{fig::BBdesgin} (rooted at $x_{11}=1$) denotes the solution set with camera $1$ is encoded as an I-frame.
The sub-tree of $x_{11}=0, x_{22}=0$ is drawn by dash lines if the optimal solution of problem~\eqref{eq::IframeSelectionSubProblem} cannot be found in this sub-tree (determined by the estimation of lower bound) so that it is not required to be traversed.
The branching phase will keep working until leaf (all variables in $\mathbf{X}$ are determined) or being bounded, where the bounding policy is described in the following subsection.

Although the branching scheme seems trivial (traverse through all the combinations of ${x_{ii},i \in V}$), different branching policy will lead to different converge speed.
For example, if camera $i$ and camera $j$ are high correlated with each other, it is clearly that letting both $x_{ii}$ and $x_{jj}$ equal to $1$ is not a good choice.
Therefore, the branch with $x_{ii}=1$ and $x_{jj}=1$ should have less priority to be traversed.
Due to this reason, we proposed a heuristic approach to determine which branch has the highest priority to be traverse for the sake of getting the optimal solution within few iterations.
The branching metric is thus defined as:
%\begin{equation}
%\frac{1}{|V|-|\tilde{X}^t|+1}+\frac{1}{lb(\tilde{X}^t)}.
%\end{equation}
%
\subsubsection{Overall Algorithm}
Based on the ideas described in the previous subsections, we can now present how the overall BB algorithm works.
The overall algorithm is now shown in algorithm~\ref{alg::bbAlgorithm}.
\IncMargin{1em}
\begin{algorithm}[]
 \SetAlgoLined
 \SetKwInOut{Input}{Input}\SetKwInOut{Output}{Output}
 \Input{Cameras set $V$ and cost matrix $\mathbf{H}$~\eqref{eq::bbCostMatrix}}
 \Output{Optimal I-frame cameras set}
 \BlankLine
 Initialize best cost = $\infty$, $t \gets 0$\\
 \While{}
 {
 	$t \gets t+1$ \\
  	\eIf{}
  	{
 	}
 	{	
 	}
 }
 \caption{\label{alg::bbAlgorithm}Branch-and-bound scheduling algorithm}
\end{algorithm}
\DecMargin{1em}
%
\subsubsection{Complexity Analysis}

{\color{red}Try to analyze the time complexity of BB algorithm. }
%
\subsection{Graph Approximation}
{\color{red}Describe our proposed heuristic I-frame selection algorithm based on minimum weight dominating set.}
%