\subsection{Multi-Camera Network Scenario}
\label{sec::clusterStructure}
We consider a scenario with a set of cameras ${\mathcal{S} = \{1,2,\cdots,|\mathcal{S}|\}}$
deployed over a city for gathering multimedia data from intersections.
A base station with limited radio resource (e.g. time slots in TDMA or resource
block in OFDMA) is then deployed in the city for collecting data from those cameras
through wireless communication links.
Data collected from individual cameras is to be periodically transmitted back to the
base station for forwarding to a back-end server.
Since the base station has limited radio resource allocated for multi-camera
networks, effective resource allocation scheme might be considered to support
transmission from cameras.
%
Therefore, we consider a \emph{two-hop relay} communication model for uplink
transmission for the sake of exploiting spatial reuse as shown in
Figure~\ref{fig::sysModel}.
That is, cameras are divided into several clusters and each cluster will select
one camera to act as cluster head for relaying data of its member to the base
station.
In such a two-tier cluster communication model, cluster heads are allocated
\emph{orthogonal radio resource} (tier-1 radio resource) for direct communicating
with the base station.
However, cameras belong to different clusters can share the same resource (tier-2
radio resource) for concurrently forwarding their data to its serving cluster head,
and hence inter-cluster interference must be properly controlled to ensure
successfully data reception by all cluster heads. 
Within each cluster, in order to prevent from intra-cluster interference, only one
transmission is active at one time instant.
%

For each transmission link, to ensure the transmitted data $D$ can be successfully
received, we require the allocated time $T$, bandwidth $W$, and link SINR $\gamma$
be sufficient as follows:
\begin{equation}
\label{eq::capaciy}
D \leq TW\log_2\left(1+\frac{1}{\Gamma}\gamma\right),
\end{equation}
where $\Gamma$ is a constant to model the gap between the achievable data rate
of the selected modulation/coding scheme and the theoretic Shannon channel
capacity~\cite{MQAM}.
To ensure energy-efficient yet reliable correlation-aware communications, therefore,
the number of clusters and the subset of machines to schedule in the same slot and
their transmission powers need to be optimally determined for the given amount of
radio resource.
We describe in the following the proposed formulation for joint optimization of
{\em cluster formation, transmission scheduling, and power control.}

\begin{figure}
\centering
\includegraphics[width=0.9\columnwidth]{Figures/model}
\caption{\label{fig::sysModel}System model for clustered communication}
\end{figure}