\subsection{Proposed P-frame Scheduling Algorithm}
\label{sec::PFrameScheduling}
%{\color{red}Describe our proposed P-frame scheduling algorithm based on a greedy scheduling metric.}
\begin{figure}
\begin{center}
\includegraphics[width=0.95\columnwidth]{Figures/PscheOverview.pdf}
\caption{\label{fig::pScheOverview}Demonstration of transmission schedule}
\end{center}
\end{figure}
As we demonstrate in figure~\ref{fig::subProbProcedure} before, the \emph{P-frame scheduling algorithm} is executed after a portion of cameras are selected to encode their images as I-frames.
Therefore, all P-frame cameras (i.e. cameras that are not be selected as I-frame cameras) can find the most correlated I-frame cameras in polynomial time when the I-frame cameras set is given.
Our algorithm thus make all P-frame cameras join the group of their ``most correlated'' I-frame camera.
Time slots will be allocated group by group, starting with the I-frame camera in that group and followed by several P-frame cameras.
The order for P-frame cameras is determined based on a greedy approach, that is, a camera that can make more other P-frames cameras to be able to overhear its transmission will be scheduled first.
The reason why we use this metric is that the later scheduled P-frame cameras might have the possibility to reference from the previous P-frame camera instead of the first scheduled I-frame camera.
However, referencing from a P-frame camera will increase the complexity of problem formulation and hence it is leave as a future work of this thesis.
Besides, we also note that the greedy P-frame scheduling metric presented above can be changed for fitting into application requirements (i.e. camera which is more delay sensitive to be scheduled first).

Figure~\ref{fig::pScheOverview} shows an example of the determined transmission schedule.
Suppose that camera $4$ and $7$ are selected to be I-frame cameras by algorithm~\ref{alg::bbAlgorithm} or~\ref{alg::minWeightDS}, and cameras $1,2,3$ are more correlated with camera $4$ and cameras $5,6$ are more correlated with camera $7$.
Therefore, cameras $1,2,3$ will join the group of camera $4$ while cameras $5,6$ will join the group of camera $7$.
Since time slots are allocated group by group, assume that the group of camera $4$ is scheduled first, then slot $1$ will be allocated for camera $4$ since I-frame camera should be scheduled first so that the later P-frame cameras are able to overhear its transmission.
To proceed, time slots $2,3,4$ will be allocated for the member cameras of camera $4$, and which member camera is scheduled at which time slot is based on the greedy P-frame scheduling metric.
After all member cameras of camera $4$ is scheduled, the next group will continued to be scheduled.
That is, slot $5$ is for the transmission of camera $7$ (the I-frame camera of the next group), and slots $6,7$ are for the member cameras of camera $7$.
Our P-frame scheduling algorithm will keep working until all cameras groups are scheduled.
%
%\begin{figure}
%\begin{center}
%\includegraphics[width=0.95\columnwidth]{Figures/referenceStructure2.pdf}
%\caption{\label{fig::encodeOrder2}Extended Coding structure of a group of pictures}
%\end{center}
%\end{figure}
%
%As we describe in chapter~\ref{sec::iFrameSelectionSubProb}, the set of cameras that encode their image as an I-frame can be determined through the proposed branch-and-bound algorithm.
%Therefore, it is clearly that all the remaining P-frame cameras can decide which camera in the I-frame cameras set is the best one to reference from while encoding their image.
%However, the schedule for those P-frame cameras is still undetermined yet, and hence we discuss in this chapter that how to determine the schedule of those P-frame cameras.
%For the P-frame cameras scheduling, we extend the coding structure of a group of pictures from figure~\ref{fig::encodeOrder} to figure~\ref{fig::encodeOrder2}.
%That is, a P-frame camera can now reference from a previous scheduled I-frame camera or P-frame camera, depends on which one is more correlated.
%However, if a P-frame camera $i$ aims to reference from another P-frame camera $j$, camera $i$ must be able to reconstruct the image of camera $j$, which means that camera $i$ also needs to overhear the transmission of the reference frame of camera $j$.
%For example, suppose that the coding structure is shown in figure~\ref{fig::encodeOrder2}, camera $4$ must both overhear the transmission of camera $1$ and $3$ if it aims to reference from the image of camera $3$.
%