\section{Protocol Design for Correlated Data Gathering in Wireless Multi-Camera Networks}
\label{sec::protocolDesign}
{\color{red} Present how our proposed correlated data gathering scheme is applied to real-world multi-camera networks.}
%
\subsection{Modified Motion Estimation Technique}
%
\begin{figure}
\begin{subfigure}[b]{\columnwidth}
\begin{center}
\includegraphics[width=0.95\columnwidth]{Figures/motionEstimation.pdf}
\caption{\label{fig::originalME}Conventional motion estimation technique}
\end{center}
\end{subfigure}
%
\begin{subfigure}[b]{\columnwidth}
\begin{center}
\includegraphics[width=0.95\columnwidth]{Figures/modifiedMotionEstimation.pdf}
\caption{\label{fig::modifiedME}Modified motion estimation technique}
\end{center}
\end{subfigure}
\caption{\label{fig::originalAndModifiedME}Demonstration motion estimation technique in multiview video coding}
\end{figure}
%
\begin{figure}
\begin{center}
\includegraphics[width=0.95\columnwidth]{Figures/biased.pdf}
\caption{\label{fig::biased}Experiment result of different biased pixels}
\end{center}
\end{figure}
{\color{red} Explain conventional motion estimation technique and our modified motion estimation in JMVC.}
As we mentioned in the previous chapter, when two cameras are observing the same area but having different sensing direction, then the collected image might just be a shifted image from the other one.
Therefore, it motivates us to modify the conventional motion estimation technique so that we can have a larger possibility to find a correlated macroblock under a given search range.
The idea of our modified motion estimation technique is shown in figure~\ref{fig::modifiedME}, where we shift the search region by the value $\kappa$ estimated from equation~\eqref{eq::biasedPixels}.
Figure~\ref{fig::biased}