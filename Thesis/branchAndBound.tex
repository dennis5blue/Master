\section{Proposed Branch and Bound Scheduling Algorithm}
\label{sec::bbAlgorithm}
\begin{figure*}
\begin{center}
\includegraphics[width=0.95\textwidth]{Figures/BBdesign.pdf}
\caption{\label{fig::BBdesgin}Enumeration tree of branch-and-bound algorithm}
\end{center}
\end{figure*}
Instead of transforming problem~\eqref{eq::formulation} into a stochastic estimation problem and solving it through the CE method, we also refer to the branch-and-bound method for finding the optimal transmission schedule.
Note that all the decision variables $\mathbf{X}, \mathbf{Y}$ in problem~\eqref{eq::formulation} belongs to a positive integer; therefore, it is clearly that the optimal solution can be found if we search through the feasible set.
That is, the solution set of problem~\eqref{eq::formulation} can be modelled as an enumeration tree.
However, directly traversing the whole tree would lead to very high complexity, and hence some policy must be included for reducing the complexity.
One possible method without losing the optimality is through the branch-and-bound (BB) algorithm~\cite{BB} as presented in this section.
The main idea of BB algorithm is that if one sub-tree has no possibility to include the optimal solution, then it is not required to be traverse any more.
Therefore, a proper policy to ``bound'' branches must be well designed for the efficiency of BB algorithm, and we introduce the ``branching'' and ``bounding'' policy in the followings.
%
\subsection{Branching Policy}
The reason why BB algorithm can be applied to solve discrete and combination optimization problems is that the candidate solutions of these problems can be thought of as a rooted tree where the full solutions set is the root of this tree.
The branching phase is thus designed for separating the solution set into several subsets.
For example, as shown in figure~\ref{fig::BBdesgin}, we divide the solution set by fixing some elements in $\mathbf{X}$.
That is, the left part of figure~\ref{fig::BBdesgin} (rooted at $x_1=1$) denotes the solution set with the first scheduled camera being camera $1$.
Note that the sub-tree of $x_1=1, x_2=1$ is drawn by dash lines, since this sub-tree is not a feasible solution set (multiple cameras scheduled in the same time slot) so that it is not required to be traversed.
The branching phase will keep working until leaf (all variables in $\mathbf{X}$ are determined) or being bounded, where the bounding policy is described in the following subsection.
\begin{equation}
\frac{1}{|V|-|\tilde{X}^t|+1}+\frac{1}{lb(\tilde{X}^t)}.
\end{equation}
%
\subsection{Bounding Policy}
Before introducing the bounding policy, we first propose an algorithm to estimate the lower bound of cost at each sub-tree.
To proceed, suppose we are now at the $t^{th}$ iteration of BB algorithm, for which the cost matrix of this iteration is defined as:
\begin{equation}
\mathbf{H}^t = \left[ \begin{array}{cccc}
\infty &h_{12}^t &\cdots &h_{1N}^t \\
h_{21}^t &\infty &\cdots &h_{2N}^t \\
\vdots &\vdots &\vdots &\vdots \\
h_{N1}^t &h_{N2}^t &\cdots &\infty
\end{array} \right],
\label{eq::bbCostMatrix}
\end{equation}
where $h_{ij}^t$ denotes the amount of bits that is required to be transmitted if camera $i$ overhears all the cameras scheduled before camera $j$.
The lower bound of cost below this branch can now be calculated through $\mathbf{H}^t$ as presented in algorithm~\ref{alg::lbEstimation}.
%
\IncMargin{1em}
\begin{algorithm}[]
 \SetAlgoLined
 \SetKwInOut{Input}{Input}\SetKwInOut{Output}{Output}
 \Input{Cost matrix $\mathbf{H}^t$ and the determined partial schedule $\tilde{X}^t$}
 \Output{Estimated lower bound $LB(\mathbf{H}^t,\tilde{X}^t)$}
 \BlankLine
 $LB(\mathbf{H}^t,\tilde{X}^t)$ $\gets$ the exact cost for the determined partial schedule $\tilde{X}^t$ \\
 \For{$i \in \tilde{X}^t$}
 {
 	Set the $i^{th}$ column of $\mathbf{H}^t$ to $\infty$ \\
 	Set the $i^{th}$ row of $\mathbf{H}^t$ to $\infty$ \\
 }
 Sort all elements in $\mathbf{H}^t$ as ascending order \\
 $LB(\mathbf{H}^t,\tilde{X}^t)$ $\gets$ $LB(\mathbf{H}^t,\tilde{X}^t)+$smallest $N-|\tilde{X}^t|$ elements of $\mathbf{H}^t$ \\
 \caption{\label{alg::lbEstimation}Estimation for lower bound}
\end{algorithm}
\DecMargin{1em}
%
The reason why we need to set some rows and columns to infinity is that as long as a camera is scheduled, it cannot be scheduled again.
\begin{equation}
\sum_{i\in V\setminus \tilde{X}^t} \underset{j \in V\setminus \tilde{X}^t}{\min}\text{ }h_{ij}^{t} - 
\underset{i\in V\setminus \tilde{X}^t}{\max} \text{ } \underset{j \in V\setminus \tilde{X}^t}{\min}\text{ }h_{ij}^{t}.
\end{equation}
%
\subsection{Overall Algorithm}
Based on the ideas described in the previous subsections, we can now present how the overall BB algorithm works.
To start, we first define an initial cost matrix $\mathbf{H}^0$ to measure the cost (amount of bits needed to be transmitted) for scheduling camera $j$ after camera $i$ for all $i,j \in V$ as:
\begin{equation}
\mathbf{H}^0 = \left[ \begin{array}{cccc}
\infty &h_{12}^0 &\cdots &h_{1N}^0 \\
h_{21}^0 &\infty &\cdots &h_{2N}^0 \\
\vdots &\vdots &\vdots &\vdots \\
h_{N1}^0 &h_{N2}^0 &\cdots &\infty
\end{array} \right],
\label{eq::initCostMatrix}
\end{equation}
where $h_{ij}^0$ is the amount of bits that is needed to be transmitted if camera $i$ overhears all the regions ${s_{jm}, m=1,2,\cdots,M}$ of camera $j$.
Therefore, $h_{ij}^0$ can be obtained by letting ${U_i^0=\{ j|\tau C_{ji} \geq H(F_j)\}}$, and $D_i^0 = \{s_{km}\in \mathcal{S}_k | k \in U_i^0\}$.
In this way,
\begin{align}
h_{ij}^0 &= H(F_i|D_i^0) \nonumber \\
         &= \sum_{m=1}^M \alpha_{im}\Delta r_{im} (1-\beta_{im}^0), \forall i,j \in V, i \neq j.
\end{align}
Note that we set the diagonal element of $\mathbf{H}^0$ to infinity due to the reason that all cameras only transmit their data once.
The overall algorithm is now shown in algorithm~\ref{alg::bbAlgorithm}.
\IncMargin{1em}
\begin{algorithm}[]
 \SetAlgoLined
 \SetKwInOut{Input}{Input}\SetKwInOut{Output}{Output}
 \Input{Relative area vector $A_i$ and rate vector $R_i$ of each camera $i \in V$}
 \Output{Optimal transmission schedule}
 \BlankLine
 Initialize cost matrix $\mathbf{H}^0$ by~\eqref{eq::initCostMatrix}\\
 Initialize best cost = $\infty$, $t \gets 0$\\
 Put $x_1 = 1$, $x_2 = 1$, $\cdots$, $x_N=1$ into branch queue\\
 \While{Branch queue $\neq$ NULL}
 {
 	$t \gets t+1$ \\
 	Update cost matrix $\mathbf{H}^t$ for this branch\\
  	\eIf{is leaf}
  	{
  		\If{cost<best cost}
  			{best cost = cost\\}
 	}
 	{
 		Estimate lower bound through algorithm~\ref{alg::lbEstimation} \\
		\eIf{estimated lower bound > best cost}
		{
  			prune this branch\\
 		}
 		{
 			keep branching\\
 		} 	
 	}
 }
 \caption{\label{alg::bbAlgorithm}Branch-and-bound scheduling algorithm}
\end{algorithm}
\DecMargin{1em}
%
\subsection{Complexity Analysis}