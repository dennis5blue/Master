\section{Evaluation Results}
\label{sec::evaluation}
\begin{figure}
\begin{center}
\includegraphics[width=0.95\columnwidth]{Figures/cityView}
\caption{\label{fig::cityView}City and camera view generated by~\cite{Suicidator,Blender}}
\end{center}
\end{figure}
In this section, we present evaluation results for the proposed scheduling approaches.
In order to evaluate the proposed approach for transmission scheduling, we resort to a 3D modeling software~\cite{Suicidator,Blender} for rendering ``semi-realistic'' city views as the data source.
An example of rendered city is shown in figure~\ref{fig::cityView}.
We then deployed 10 cameras with different view angles in the ``virtual'' city to capture different but correlated views of the city.

Each camera deployed is configured to take the city snapshot as a $1280 \times 720$ HD image.
To find the encoding rate vector $R_i$ of each camera, we divide the image into regions and use a multi-view video coding reference software~\cite{JMVC} to process those regions (treated as an I-frame) of each camera's view.
Afterwards, we experimentally check if two different regions are correlated with each other by using multi-view encoding for each pair of regions.
%
\subsection{Evaluation of Different Scheduling Algorithms}
\begin{figure}
\begin{center}
\includegraphics[width=0.95\columnwidth]{Figures/txBytes.pdf}
\caption{\label{fig::evaScheduling}Comparison of different scheduling algorithms}
\end{center}
\end{figure}
%
\begin{figure}
\begin{center}
\includegraphics[width=0.95\columnwidth]{Figures/txFrame.pdf}
\caption{\label{fig::evaTxFrame}Transmitted frame after overhearing nearby transmissions}
\end{center}
\end{figure}
In this subsection, we present the evaluation results of cross entropy and branch-and-bound scheduling algorithm for overhearing source coding.
Through the overhearing source coding mechanism, each camera can leverage the transmitted frame of previous scheduled cameras for reducing the amount of bits needed to be delivered.
That is, some regions are no longer necessary to be transmitted after processing the overheard frames.
Based on this idea, the cross entropy and branch-and-bound algorithm is then applied to determine the transmission schedule, and we can obtain $22\%$ transmission bits reduction for both two algorithms, which is better than $11\%$ (greedy scheduling algorithm) as shown in figure~\ref{fig::evaScheduling}.
Note that the branch-and-bound algorithm can perform better than the cross entropy algorithm since it is promised to obtained the optimal solution.

Figure~\ref{fig::evaTxFrame} further demonstrates the original frame of an arbitrary camera and its transmitted frame.
It can be shown that some regions become blank since they are correlated to the transmissions of previous cameras.
Therefore, the amount of bits for encoding this frame can be reduced and the radio resource can be utilized in a more efficient way.
%
\subsection{Evaluation of Convergence}
\begin{figure}
\begin{center}
\includegraphics[width=0.95\columnwidth]{Figures/converge.pdf}
\caption{\label{fig::evaBBConvergence}Convergence analysis of BB algorithm}
\end{center}
\end{figure}
In addition to getting the optimal transmission schedule, we are also interested in the time complexity of the branch-and-bound algorithm.
Figure~\ref{fig::evaBBConvergence} shows the upper bound and lower bound of the BB algorithm during each iteration.
It can be seen that the BB algorithm stops (the optimal solution is obtained) at about  $3000$ iterations, which means that the time complexity is reduced $99\%$ compared to the brute force searching method.
Besides, there still exists a performance gap between the BB algorithm and brute force algorithm when $3000$ iterations are done.
This gap even becomes larger when the number of iterations is only $100$.
Therefore, we can claim that our proposed branch-and-bound algorithm can obtain the optimal scheduling solution in an efficient way.