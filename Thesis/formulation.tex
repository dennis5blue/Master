\section{Transmission Scheduling under Overhearing Source Coding}
\label{sec::OSC}
Denote ${V=\{ 1,2,\cdots,N \}}$ as the set of cameras deployed for data gathering in a city, where camera $i$ will produce image $F_i$ containing $\Delta$ pixels.
To process and transmit the image observed by each camera, we assume that $F_i$ can be decomposed into $M$ regions ${s_{im}}$, where ${m = 1,2,\cdots,M}$ and we denote $\mathcal{S}_i$ as the set of such regions.
For each region ${s_{im} \in \mathcal{S}_i}$, we denote by ${A_i = [\alpha_{i1}, \alpha_{i2}, \cdots, \alpha_{iM}]}$ as the relative area vector such that $\alpha_{im}$ is the relative area of frame $F_i$ dedicated to the region $s_{im}$.
Note that ${\sum_{s_{im} \in \mathcal{S}_i} \alpha_{im} = 1}$ since the sum of every regions must equal to the original frame $F_i$.
Besides, we also define the rate vector ${R_i = [r_{i1}, r_{i2}, \cdots, r_{iM}]}$ where $r_{im}$ is the rate (in bits per pixel) for region $s_{im} \in \mathcal{S}_i$.
Based on the above assumptions, we show in the following our motivation and how the resource allocation solution can be obtained through overhearing source coding.
%
\subsection{Motivation}
\begin{figure}
\begin{center}
\includegraphics[width=0.95\columnwidth]{Figures/motivation.pdf}
\caption{\label{fig::motivation}Motivation of data-centric data gathering}
\end{center}
\end{figure}
In multi-camera networks, image collected from different cameras might eventually produce correlated data streams.
It is clearly that the amount of data gathered from all cameras within the network is increased together with the number of cameras.
However, as shown in figure~\ref{fig::motivation}, useful data does not increased as much as the data quantity, which means that there exists a portion of data which is not required to be transmitted, and hence motivates us to investigate the idea of correlated data gathering through overhearing source coding.
%
\subsection{Problem Formulation}
Since the performance of overhearing source coding depends a lot on the sequence of transmission schedule, the goal of our problem formulation is to determine the best schedule so that the later scheduled camera can leverage the image collected by the set of previous scheduled cameras for reducing its amount of data.
To start, assume that every cameras need to deliver its image to the aggregator once, and the number of available time slots is equivalent to number of cameras $N$.
We now denote the vector ${\mathbf{X} = [x_1, x_2, \cdots, x_N]}$, ${1 \leq x_i \leq N}$, as the index of time slot allocated for all cameras such that $x_i = k$, ${1 \leq k \leq N, k \in \mathbb{Z}}$, if camera $i$ is scheduled at the $k^{th}$ time slot.
Note that we require ${|x_i - x_j| > 0}$, ${\forall i,j \in V, i \neq j }$ since only one camera can transmit its data within the same time slot.
For each camera $i$, it is not necessary to transmit all regions $s_{im} \in \mathcal{S}_i$.
Therefore, we further define the matrix ${\mathbf{Y} = [y_{im}]_{N \times M}}$ as an indicator matrix such that
\begin{equation*}
\left\{ \begin{array}{ll}
y_{im} = 1, &\text{ if region $s_{im}$ of image $F_i$ has been delivered} \\
                   &\text{ to the aggregator,} \\
y_{im} = 0, &\text{ if region $s_{im}$ of image $F_i$ has not been} \\
                   &\text{ delivered to the aggregator.}
\end{array} \right.
\end{equation*}

To proceed, we can define the subset of cameras that transmit before camera $i$ as
\begin{equation}
W_i = \{ j \in V | x_j < x_i \}.
\label{eq::prevSet}
\end{equation}
For the sake of determining the set of overhearing cameras, we do not assume a constant, given overhearing distance for all machines.
Instead, we take a general approach by assuming the set of cameras that can overhear a transmission to be determined based on the amount of transmitted data and the channel capacity.
Specifically, for camera $i$ to overhear transmission from camera $j \in W_i$, the amount of data to overhear cannot be larger than the maximum channel capacity $C_{ji}$ multiplied by the time slot duration $\tau$.
In this way, the set of cameras $U_i$ that camera $i$ can overhear can be expressed as follows:
\begin{equation}
U_i = \{ j \in W_i | \tau C_{ji} \geq H(F_j|D_j) \},
\label{eq::overSet}
\end{equation}
where $\tau$ is the duration of time slot and $C_{ji}$ is the achievable rate from camera $j$ to camera $i$ defined as:
\begin{equation}
C_{ji} = W \log_2 \left( 1+\frac{1}{\Gamma} \frac{p_j G_{ji}}{WN_0} \right),
\label{eq::capacity}
\end{equation}
with $p_j$ being the transmission power, $G_{ji}$ being the channel gain, $W$ begin the channel bandwidth, $N_0$ being the background noise, and $\Gamma$ being the Shannon gap as introduced in~\cite{MQAM}.
The set $D_j$ in~\eqref{eq::overSet} denotes the set of regions that can be obtained at camera $j$, which can be defined as:
\begin{equation}
D_j = \{ k \in U_j, s_{km} \in \mathcal{S}_k | y_{km} = 1 \}.
\label{eq::overRegionSet}
\end{equation}
Note that for the special case when the channel gain $G_{ji}$ between camera $j$ and camera $i$ is uniquely determined by the distance $d_{ji}$ between them (e.g. no fading), transmission of camera $j$ can be overheard by cameras that are inside the circle of radius $d_{j0}$ around camera $j$.
After overhearing transmissions from the set $D_i$, camera $i$ can perform dependent source coding for removing redundant information from its data.
The required amount of bits for encoding $F_i$ can now be reduced as follows:
\begin{align}
H(F_i|D_i) &= \sum_{m=1}^{M} \alpha_{im} \Delta r_{im} y_{im}, \nonumber \\
           &\geq \sum_{m=1}^{M} \alpha_{im} \Delta r_{im} (1-\beta_{im}),
\label{eq::conditionalEntropy}
\end{align}
where $\beta_{im} = 1$ if region $s_{im} \in \mathcal{S}_i$ is correlated to some regions within the set $D_i$, and hence $s_{im}$ is not necessary to be transmitted.

Recall that the objective for the scheduling problem is to minimize the total amount of bits that needed to be delivered.
The overall problem formulation can now be written as follows:
\begin{equation*}
\underset{\mathbf{X}, \mathbf{Y}}{\min} \sum_{i=1}^{N} H(F_i|D_i),
\end{equation*}
subject to
\begin{align}
&W_i = \{ j \in V | x_j < x_i \}, &\forall i \in V, \nonumber \\
&U_i = \{ j \in W_i | \tau C_{ji} \geq H(F_j|D_j) \}, &\forall i \in V, \nonumber \\
&D_i = \{ j \in U_i, s_{jm} \in \mathcal{S}_j | y_{jm} = 1 \}, &\forall i \in V, \nonumber \\
&|x_i - x_j| > 0, &\forall i,j \in V, i \neq k, \nonumber \\
&y_{im} \geq 1-\beta_{im}, &\forall i \in V, 1 \leq m \leq M, \nonumber \\
&1 \leq x_i \leq N, x_i \in \mathbb{Z}, &\forall i \in V, \nonumber \\
&y_{im} = \{0,1\}, &\forall i \in V, 1 \leq m \leq M.
\label{eq::formulation}
\end{align}
Since the determination of $\mathbf{X}$ and $\mathbf{Y}$ in problem~\eqref{eq::formulation} is $\mathcal{NP}$-hard, we describe in the following how we solve this problem through cross entropy and branch-and-bound approach.
%
\subsection{Complexity Analysis}