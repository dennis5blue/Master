\section{Introduction}
\label{sec::introduction}

Multimedia machine-to-machine (M2M) communications have been typically envisioned to
leverage the infrastructure of the mainstream cellular networks for enabling
{\em ambient intelligence} in the Internet of things.
For instance, cameras deployment in the traffic surveillance systems gives the
possibility to automatically analyze the urban traffic density~\cite{Kapsch,Traficon,Citilog}.
%
Unlike conventional wireless sensor networks with dedicated network deployment,
one challenging problem for those multimedia M2M applications is to share the
precious cellular radio resource with existing cellular users.
%
In addition, multimedia data often generates larger packets size and hence the
limited wireless radio resource might become insufficient to support the
{\em real-time transportation of the multimedia data}.

Fortunately, since the deployment of cameras often has overlapped region of interest, to
be able to transport a large amount of data from cameras to the back-end server,
correlated data gathering has been considered as one effective technology for leveraging
data correlation.
On the one hand, multiple cameras observing the same area from different point of view
will produce correlated video streams. 
On the other hand, multiview video coding can allow us to reduce the packet size
required to encode frames collected from different neighborhood cameras.
Therefore, in this paper, we exploit the idea of correlated data gathering in wireless
multimedia camera networks for the sake of reducing radio resource as well as increasing
the amount of collected data.
%
While correlated data gathering has been popularly investigated in the literature, most
related work has modeled the correlation between cameras from a theoretic perspective
using their predefined position.
As we describe in Section~\ref{sec::relatedWork}, such an assumption fails to consider
the realistic correlation level in the multimedia devices.
Hence, the theoretic correlation model may lead to sub-optimal solution in the context
of multimedia M2M communications with {\em tight radio resource constraints.}

In this paper, we investigate the problem of correlated data gathering from a set of
cameras deployed in a city.
It is required that these cameras periodically send back their data back to a back-end
server through direct wireless communications (e.g. LTE).
To exploit data correlation among cameras, we allow cameras to only transmit part of
its collected frames to the back-end server.
We take the communication constraint into consideration to determine the transmission
schedule of cameras as well as the data required to be delivered such that more useful
data can be collected under limited wireless radio resource.
Unlike related work on similar problems, our goal is not to maximize the link quality
(sum data rate) or the amount of cameras that can be served.
Instead, realizing the fact that cameras are typically deployed to perform one task
collectively, we {\em prioritize data over camera} and allow cameras with less
important data to be dropped from support under resource limitation.
%We present in Section~\ref{sec::framework} and Section~\ref{sec::problemFormulation} our communication model and problem formulation.
%

%To solve the joint optimization problem, we present in Section~\ref{sec::clustering}~and~\ref{sec::scheduling} how we solve the formulated problem through decoupling into the {\em cluster formation} sub-problem and the {\em joint transmission scheduling and power control} sub-problem. 
%Based on the evaluation for a semi-realistic camera surveillance network in Section~\ref{sec::results}, the proposed approach can effectively provide resource-efficient transmission.
%Simulation results show that compared to the baseline approaches, the proposed approach can result in minimum power consumption while keeping the fidelity of received data in those multimedia M2M applications, and finally, Section~\ref{sec::conclusion} concludes our work.