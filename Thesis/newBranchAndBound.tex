\section{Proposed Scheduling Algorithm}
\label{sec::schedulingAlgorithm}
%\begin{figure*}
\begin{figure}
\begin{center}
\includegraphics[width=0.95\columnwidth]{Figures/BBdesignIframeSelection.pdf}
\caption{\label{fig::BBdesgin}Enumeration tree of branch-and-bound algorithm}
\end{center}
\end{figure}
%\end{figure*}
%
\begin{figure}
\begin{center}
\includegraphics[width=0.95\columnwidth]{Figures/referenceStructure.pdf}
\caption{\label{fig::encodeOrder}Coding structure of a group of pictures}
\end{center}
\end{figure}
As we mentioned before, problem~\eqref{eq::formulation} is a $\mathcal{NP}$-hard problem, therefore, we show how to simplify and solve this problem in this chapter.
Note that we assume a P-frame can only reference from one I-frame as illustrated in equation~\eqref{eq::referenceOnlyIframe}.
According to equation~\eqref{eq::referenceOnlyIframe}, the coding structure of a group of pictures can be shown in figure~\ref{fig::encodeOrder}, where every P-frames will reference from the closest previous I-frame to reduce its encoded bits.
Consequently, problem~\eqref{eq::formulation} can be decomposed into two sub-problems, including (a) the I-frame selection sub-problem and (b) the P-frame scheduling sub-problem.
We show in the followings how we solve these two sub-problems.

\subsection{I-frame Selection Sub-Problem}
Note that the decision variables of the I-frame selection sub-problem are the diagonal elements of $\mathbf{X}$.
We aim to figure out whether ${x_{ii},i \in V}$ equals to $0$ or $1$ so that the total amount of encoded bits can be minimized, where the other scheduling and overhearing constraints are ignore during the selection of I-frames.
Suppose that camera $i$ is encoded as an I-frame, it is clearly that it requires $H(F_i)$ to encode its image.
However, if camera $i$ is encoded as a P-frame, the amount of encoded bits should depend on which I-frame it references from.
An intuitive method to determine the reference frame is to select the most correlated I-frame and the amount of encoded bits can be written as $\underset{j \in V, x_{jj}=1}{\min} H(F_i|F_j)$.
Therefore, the I-frame selection sub-problem can be written as
\begin{equation*}
\min \left(
\sum_{i=1}^{|V|} x_{ii}H(F_i) +
\sum_{i=1}^{|V|} (1-x_{ii}) \underset{j \in V, x_{jj}=1}{\min} H(F_i|F_j) \right),
\end{equation*}
subject to
\begin{align}
&x_{ii} = \{0,1\}, &\forall i \in V.
\label{eq::IframeSelectionSubProblem}
\end{align}

Since all the decision variables ${x_{ii},i \in V}$ belong to a binary integer, it is clearly that the optimal solution can be found if we search through the feasible set.
That is, the solution set of problem~\eqref{eq::IframeSelectionSubProblem} can be modeled as an enumeration tree.
However, directly traversing the whole tree would lead to very high computational complexity, and hence some policy must be included for reducing the complexity.
One possible method without losing the optimality is through the branch-and-bound (BB) algorithm~\cite{BB} as presented in this chapter.
The main idea of the BB algorithm is that if one sub-tree has no possibility to include the optimal solution, then it is not required to be traversed any more.
Therefore, a proper policy to ``bound'' branches must be well designed for the efficiency of the BB algorithm, and we introduce the ``branching'' and ``bounding'' policy in the followings.
%
\subsubsection{Branching Policy}
The reason why BB algorithm can be applied to solve discrete and combination optimization problems is that the candidate solutions of these problems can be thought of as a rooted tree where the full solutions set is the root of this tree.
The branching phase is thus designed for separating the solution set into several subsets.
For example, as shown in figure~\ref{fig::BBdesgin}, we divide the solution set by fixing some elements of ${x_{ii},i \in V}$.
That is, the right hand side of figure~\ref{fig::BBdesgin} (rooted at $x_{11}=1$) denotes the solution set with camera $1$ is encoded as an I-frame.
Note that the sub-tree of $x_{11}=1, x_{22}=1$ is drawn by dash lines if the optimal solution of problem~\eqref{eq::IframeSelectionSubProblem} cannot be found in this sub-tree (determined by the estimation of lower bound) so that it is not required to be traversed.
The branching phase will keep working until leaf (all variables in $\mathbf{X}$ are determined) or being bounded, where the bounding policy is described in the following subsection.
%\begin{equation}
%\frac{1}{|V|-|\tilde{X}^t|+1}+\frac{1}{lb(\tilde{X}^t)}.
%\end{equation}
%
\subsubsection{Bounding Policy}
Before introducing the bounding policy, we first propose an algorithm to estimate the lower bound of cost at each sub-tree.
To proceed, we first denote the cost matrix for the I-frame selection sub-problem as:
\begin{equation}
\mathbf{H} = \left[ \begin{array}{cccc}
H(F_1) &H(F_1|F_2) &\cdots &H(F_1|F_{|V|}) \\
H(F_2|F_1) &H(F_2) &\cdots &H(F_2|F_{|V|}) \\
\vdots &\vdots &\vdots &\vdots \\
H(F_{|V|}|F_1) &H(F_{|V|}|F_2) &\cdots &H(F_{|V|})
\end{array} \right].
\label{eq::bbCostMatrix}
\end{equation}
The lower bound of cost below a branch can now be calculated through $\mathbf{H}$ as presented in algorithm~\ref{alg::lbEstimation}.
%
\IncMargin{1em}
\begin{algorithm}[]
 \SetAlgoLined
 \SetKwInOut{Input}{Input}\SetKwInOut{Output}{Output}
 \Input{Cost matrix $\mathbf{H}^t$ and the determined partial schedule $\tilde{X}^t$}
 \Output{Estimated lower bound $LB(\mathbf{H}^t,\tilde{X}^t)$}
 \BlankLine
 $LB(\mathbf{H}^t,\tilde{X}^t)$ $\gets$ the exact cost for the determined partial schedule $\tilde{X}^t$ \\
 \For{$i \in \tilde{X}^t$}
 {
 	Set the $i^{th}$ column of $\mathbf{H}^t$ to $\infty$ \\
 	Set the $i^{th}$ row of $\mathbf{H}^t$ to $\infty$ \\
 }
 Sort all elements in $\mathbf{H}^t$ as ascending order \\
 $LB(\mathbf{H}^t,\tilde{X}^t)$ $\gets$ $LB(\mathbf{H}^t,\tilde{X}^t)+$smallest $N-|\tilde{X}^t|$ elements of $\mathbf{H}^t$ \\
 \caption{\label{alg::lbEstimation}Estimation for lower bound}
\end{algorithm}
\DecMargin{1em}
%
The reason why we need to set some rows and columns to infinity is that as long as a camera is scheduled, it cannot be scheduled again.
\begin{equation}
\sum_{i\in V\setminus \tilde{X}^t} \underset{j \in V\setminus \tilde{X}^t}{\min}\text{ }h_{ij}^{t} - 
\underset{i\in V\setminus \tilde{X}^t}{\max} \text{ } \underset{j \in V\setminus \tilde{X}^t}{\min}\text{ }h_{ij}^{t}.
\end{equation}
%
\subsubsection{Overall Algorithm}
Based on the ideas described in the previous subsections, we can now present how the overall BB algorithm works.
To start, we first define an initial cost matrix $\mathbf{H}^0$ to measure the cost (amount of bits needed to be transmitted) for scheduling camera $j$ after camera $i$ for all $i,j \in V$ as:
\begin{equation}
\mathbf{H}^0 = \left[ \begin{array}{cccc}
\infty &h_{12}^0 &\cdots &h_{1N}^0 \\
h_{21}^0 &\infty &\cdots &h_{2N}^0 \\
\vdots &\vdots &\vdots &\vdots \\
h_{N1}^0 &h_{N2}^0 &\cdots &\infty
\end{array} \right],
\label{eq::initCostMatrix}
\end{equation}
where $h_{ij}^0$ is the amount of bits that is needed to be transmitted if camera $i$ overhears all the regions ${s_{jm}, m=1,2,\cdots,M}$ of camera $j$.
Therefore, $h_{ij}^0$ can be obtained by letting ${U_i^0=\{ j|\tau C_{ji} \geq H(F_j)\}}$, and $D_i^0 = \{s_{km}\in \mathcal{S}_k | k \in U_i^0\}$.
In this way,
\begin{align}
h_{ij}^0 &= H(F_i|D_i^0) \nonumber \\
         &= \sum_{m=1}^M \alpha_{im}\Delta r_{im} (1-\beta_{im}^0), \forall i,j \in V, i \neq j.
\end{align}
Note that we set the diagonal element of $\mathbf{H}^0$ to infinity due to the reason that all cameras only transmit their data once.
The overall algorithm is now shown in algorithm~\ref{alg::bbAlgorithm}.
\IncMargin{1em}
\begin{algorithm}[]
 \SetAlgoLined
 \SetKwInOut{Input}{Input}\SetKwInOut{Output}{Output}
 \Input{Relative area vector $A_i$ and rate vector $R_i$ of each camera $i \in V$}
 \Output{Optimal transmission schedule}
 \BlankLine
 Initialize cost matrix $\mathbf{H}^0$ by~\eqref{eq::initCostMatrix}\\
 Initialize best cost = $\infty$, $t \gets 0$\\
 Put $x_1 = 1$, $x_2 = 1$, $\cdots$, $x_N=1$ into branch queue\\
 \While{Branch queue $\neq$ NULL}
 {
 	$t \gets t+1$ \\
 	Update cost matrix $\mathbf{H}^t$ for this branch\\
  	\eIf{is leaf}
  	{
  		\If{cost<best cost}
  			{best cost = cost\\}
 	}
 	{
 		Estimate lower bound through algorithm~\ref{alg::lbEstimation} \\
		\eIf{estimated lower bound > best cost}
		{
  			prune this branch\\
 		}
 		{
 			keep branching\\
 		} 	
 	}
 }
 \caption{\label{alg::bbAlgorithm}Branch-and-bound scheduling algorithm}
\end{algorithm}
\DecMargin{1em}
%
\subsubsection{Complexity Analysis}
%
\subsection{P-frame Scheduling Sub-Problem}