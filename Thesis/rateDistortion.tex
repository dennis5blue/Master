\subsection{Rate-Distortion Model}
\label{sec::rateDistortion}
In order to reduce the power consumption of cameras while keeping the quality of images,
we refer to the rate-distortion model proposed in~\cite{CorrAwareScheduling} for
reconstructing multiview images.
Suppose that the $m^{th}$ camera acquires the image $F_{m}$ that is compressed at a
rate of $R_{m}$ bits per pixel ($m=1,\cdots,|\mathcal{S}|$), and a subset of the compressed images
captured by all cameras is received at the data aggregator, which targets the
reconstruction of the full scene.
If the frame $F_{m}$ is available at the decoder, the distortion is directly dependent on the compression or the source rate.
However, if $F_{m}$ is missing at decoder, it is reconstructed from the available neighboring frames.
The overall distortion of the scene is thus expressed as
\begin{equation}
D(\mathbf{R}) = \sum_{m=1}^{|\mathcal{S}|} \frac{1}{\omega_m} D_{m}(\mathbf{R}),
\end{equation}
where $\omega_m$ represents the relative importance of the view of the $m^{th}$ camera.
The distortion $D_{m}(\mathbf{R})$ is the distortion of the $m^{th}$ view, which can be
further estimated by decomposing the frame into regions $s_j$ and we denote by
$\mathcal{S}_{m}$ as the set of such regions.
For each $s_j \in \mathcal{S}_{m}$, we denote by $\alpha (s_j)$ the relative area of the
frame dedicated to the region $s_j$, such that
${\sum_{s_j \in \mathcal{S}_{m}} \alpha (s_j) = 1}$.

Then, a mapping function $\mathbf{\Phi}_{j,m}$ describes which of the neighboring frames can contribute to the reconstruction of the region $s_j$ of the $m^{th}$ view.
That is, ${\mathbf{\Phi}_{j,m} = [\Phi_{j,m}(1) \cdots \Phi_{j,m}(|\mathcal{S}|)]}$, where ${\Phi_{j,m}(k) = 1}$ if the $k^{th}$ camera is correlated with the region $s_j$ of the frame $F_{m}$ and ${\Phi_{j,m}(k) = 0}$ otherwise.
Equipped $\mathbf{\Phi}_{j,m}$ with the above notation, the distortion becomes the sum of the distortion in each part $s_j$ of the frame:
\begin{equation}
D_{m}(\mathbf{R}) = \sum_{s_j \in \mathcal{S}_{m}} \alpha (s_j) d[\mathbf{\Phi}_{j,m} \cdot \mathbf{R}],
\label{eq::rateDis}
\end{equation}
if the view is not received, $d[R_{m}]$ otherwise.
Finally, the distortion functions in~\eqref{eq::rateDis} can be evaluated from the general expression of the R-D function of an intra-coded frame with high rate assumption:
\begin{equation}
d[R_I] = \mu_I \sigma_I^2 2^{-2R_I},
\end{equation}
where $R_I$ is the number of bits per pixels and is equal to the sum of the rates that contribute to the current region, $\sigma_I^2$ is the spatial variance of the frame and $\mu_I$ is a constant depending on the source distribution.

\begin{equation}
D_i (\mathbf{R}_1,\mathbf{R}_2,\cdots,\mathbf{R}_N) = \sum_{m=1}^{M} \alpha(f_{i,m}) d[ \sum_{j=1}^N \mathbf{\Phi}_{i,j,m}\cdot \mathbf{R}_j ]
\end{equation}

\begin{equation}
\mathbf{\Phi}_{i,j,m} = [\phi_{i,j,m}(1),\cdots,\phi_{i,j,m}(M)]
\end{equation}