\subsection{Related Work}
\label{sec::relatedWork}
As we mentioned in the previous chapters, the rise of multi-camera networks has prompted researchers to investigate challenging issues in different areas~\cite{VsnChallenges}.
The idea of multi-camera networks can be applied for a wide range of applications, for example, cameras deployed in the distributed video-based surveillance system~\cite{VideoBasedSurveillanceSystem} are responsible for collecting visual data, processing the visual data collaboratively, and transmitting the data to a data aggregator.
However, data transmission in multi-camera networks is considered to be very different from conventional scalar sensor networks.
That is, a huge amount of radio resource, for example, might be required to support a wireless surveillance system to provide real-time transmission of the image/video over wireless communication links.
Besides, the correlation of visual data is also very different from scalar data since even though two cameras are deployed at the same position, different sensing direction will cause them producing different video streams.
To optimize the system performance of multi-camera networks, related works have investigated this topic through various perspectives.
Therefore, before presenting our proposed scenario, we first introduce several related works in this chapter, and the related works are separated in the following four categories: multi-camera networks, resource allocation, visual correlation model, and overhearing source coding.
%
\subsubsection{Multi-camera Networks}
Multi-camera networks have been considered as powerful tools for various safety and security applications for different environments such as highways or subway stations~\cite{MultiCameraNetworksBook}.
Since visual data for those applications often required a huge amount of allocated radio resource for transmission, some popular research topics for the multi-camera networks include (a) how to intelligently encode distributed visual information collected by cameras and (b) how to efficiently deliver the visual data to a data aggregator.

To address the encoding problem, distributed source coding (DSC)~\cite{SlepianWolf}~\cite{WynerZiv} has been proposed for improving the coding efficiency by exploiting the spatial correlation among nearby cameras.
The idea of distributed source coding is to assume there has no communication among cameras when encoding the visual data, however, the decoder needs to jointly decode the collected video streams and hence the computation complexity is shifted from the encoder to decoder.
The author in~\cite{DVC} further based on the results in~\cite{SlepianWolf}~\cite{WynerZiv} and proposed a new paradigm for video compression through distributed video coding (DVC).
In distributed video coding, an encoder encodes individual frames independently, whereas the decoder jointly decodes the encoded frames.
Side information such as previous frames is often needed by the decoder to successfully decode all received frames.
The authors in~\cite{DVCinMVC} extended the concept of distributed video coding to multiview systems, where the side information is generated by exploiting the spatial correlation among different cameras.
Our work, on the other hand, does not incur the complexity at the DVC decoder by leveraging the inherent overhearing capability at the encoders.

In addition to referring to distributed video coding for improving the efficiency of encoding visual data, some researchers also focus on optimizing the delivery procedure of multiview video streams.
The authors in~\cite{P2Pstreaming} considered how to provide multiview video streaming over a peer-to-peer (P2P) network.
They proposed a streaming protocol such that each view of the multiview video is transmitted over an independent P2P streaming tree.
Although the streaming protocol in~\cite{P2Pstreaming} is less affected by the packet loss while delivering data, the coding efficiency can still be improved since they assume that different paths are independent with each other.
Besides, some researches focus on selective streaming for the sake of reducing bandwidth requirement during data transmission.
The work in~\cite{ClientDrivenStreaming} introduced a client-driven multiview streaming system that allows users to interactively watch 3D video.
The idea is that only the views which are required to display at the user's side are transmitted through the internet.

For exploiting redundancy among multiple surveillance cameras, the authors in~\cite{ClusteredSynopsis} propose to cluster similar activities from different cameras into a shorter video summary to improve the efficiency of browsing and searching the video.
The work, however, leaves open how such huge amount of videos can be collected through wireless communication links.
The authors in~\cite{CameraSelection} propose a different approach to reduce the amount of transmitted image data by selecting only a subset of cameras to report their information.
The goal of the selection is to find the set of images that contribute most significantly to the desired observation.
The authors in~\cite{CorrAwareScheduling} further proposed a scheduling algorithm based on the importance of data collected by each cameras, and the transmission of later scheduled cameras might be dropped due to the constraints of wireless radio resource.
Such a selective report mechanism is different from our work, where the information from all cameras are retained without loss in the overall quality of the data collected at the aggregator.
%
\subsubsection{Resource Allocation}
Problem of resource allocation for correlated network is also investigated in many literatures.
The authors in~\cite{CorrelationAwareRA} considered the problem of resource allocation (channel allocation) for correlated data sources in multi-channel multicell FDMA networks, where each source decides the uplink power to use in each channel for communicating with the base station that it belongs to.
Under the assumption of the Gaussian source and exponential correlation model, the authors allowed data sources to form source groups for joint decoding and aim to minimize the maximum distortion of all source groups through optimal power allocation and channel assignment.
It proposed a three-step method to separate the problem into inter-cell resource management, grouping of sources for joint decoding, and intra-cell channel assignment.
In~\cite{AdaptiveCrossLayerRA}, the authors considered the resource allocation problem for uplink transmission of correlated video sources, where multiple sources stream live video to the base station on a shared channel.
It aims to maximize the weighted sum of qualities of all received videos, and consists of finding optimal code choice, power assignment and packet selection.
By relating the quality of decoded video to quality-rate characteristic functions of videos, it formulates a convex optimization problem with linear constraint for solving.
%
\subsubsection{Visual Correlation Model}
The authors in~\cite{SpatialCorrelationModel} propose a spatial correlation model for cameras deployed in a neighborhood area.
In their model, the correlation of two cameras is determined by their location and the difference of their sensing direction.
In reality, however, the camera views are rather complicated and it is often insufficient to model the correlation based only on geometric information.
As demonstrated in~\cite{RealisticModel} based on H.264 multiview video coding, the discrepancy increases as the angular difference between cameras increases.
Since our work is based on real H.264 multiview coding, we do not suffer from the same problem as in~\cite{SpatialCorrelationModel}.
%
They refer to the spatial correlation model for cameras deployed in a neighborhood area proposed in~\cite{SpatialCorrelationModel}.
In this model, the correlation of two cameras is determined only by their location and the difference of their sensing direction.
In reality, however, the camera views are rather complicated and it is often insufficient to model the correlation based only on geometric information.
As demonstrated in~\cite{RealisticModel} based on H.264 multiview video coding, the discrepancy increases as the angular difference between cameras increases.
Since our work is based on real H.264 multiview coding, we do not suffer from the same problem as in~\cite{DMCPclustering,imageModelCluster}.
Instead, the image correlation model used in our work is based on actual coding of images and hence might lead to more realistic results.
%
\subsubsection{Overhearing Source Coding}
Exploiting the possibility of overhearing for data reduction is investigated in literatures, and the optimal selection and cooperation for joint encoding is considered.
The work in~\cite{imageModelCluster} exploits overhearing for visual correlation based image gathering.
Each camera in the target scenario is allowed to overhear transmission from one camera to perform different coding of its image, and the goal is to determine for each camera the node to overhear such that the energy consumption is minimized.
Similarly, the work in~\cite{BodyAreaSensorNetwork} also allows each node to overhear data from another node for data compression.
An algorithm for determining the partial orderings of transmission priorities is proposed to facilitate data compression.
The work in~\cite{InterSessionCoding} considers the problem of inter-session network coding in wireless mesh networks.
It illustrates the opportunity for machines to reduce transmission rate from overhearing and increases the network throughput.
To enable additional opportunities for packet combining and network coding, a node is allowed to increase the range over which its packets are overheard.
A joint optimization problem for determining the transmission rate vector and coding scheme is formulated and an algorithm for a simplified pairwise network coding problem is proposed.
The technique of controlling the transmission power for adjusting the number of nodes that can overhear transmission is further exploited in~\cite{BridgeMonitoring}.
In~\cite{SpatialCorrInBodySensorNetwork} discusses the application in Human body motions, which is an extension from\cite{BodyAreaSensorNetwork} to exploit more overhearing nodes.
It considers node $v_i$ to compress motion data by allowing it to overhear a set $S_i$ containing most $\kappa$ others nodes, and sends averagely $c(v_i|S_i)$ size of data.
It discusses the minimizing of transmission cost by taking use of the conditional entropy as an optimal rate, and propose a solution based on finding the minimum cost DAG for it.
It shows finding the minimum cost DAG for $\kappa > 2$ is a NP-hard problem and proposes a heuristic greedy cycle breaker algorithm for solving the problem.
Experimental results show that such approach is quite close to the optimum in the motion data application.
%