\subsubsection{H.264 Video Compression Technique}
\label{sec::H264CompressionIntro}
%
\begin{figure}
\begin{center}
\includegraphics[width=0.95\columnwidth]{Figures/baselineGOP.pdf}
\caption{\label{fig::baselineGOP}Group of pictures}
\end{center}
\end{figure}
%
The main idea of video compression in H.264 is to remove redundant video data so that the compressed file can be efficiently transmitted through the internet.
Usually, this can be done by encoding the source video stream at the transmitter side, where the encoding technique can be known as as a ``difference coding'' method.
More specifically, in order to ensure that the redundant information such as static background is not repeatedly transmitted, the video encoder (e.g. cameras in the multi-camera network) will compare the difference between the current video frame with the previous frame and perform differential encoding for the sake of removing the redundant part of the current video frame.
When encoding video frames, the three following types are defined in the standard of H.264:
\begin{itemize}
\item \textbf{I-frame}:
An I-frame is a video frame which has been encoded independently (i.e. without referencing from any other frame).
Therefore, an I-frame can be decoded at the receiver side without any help of other frames.
Due to this reason, any video streams will always start with a frame encoded as an I-frame and will have subsequent I-frames added after encoding several frames.
The interval between successive I-frames is an important issue in the H.264 video compression technique.
On one hand, I-frames are necessary for random accessing different parts of the video files since they are the only frame type which can be decoded independently.
On the other hand, encoding video frames as I-frames has the drawback that they are the largest in terms of frame size since only intra-frame redundancy can be removed for this type of frames.
\item \textbf{P-frame}:
A P-frame is a video frame that exploits preceding I or P-frame as its reference when encoding.
That is, the video encoder will perform a searching algorithm on the reference I or P-frame when encoding a P-frame.
As long as some areas are found to be unchanged between a P-frame and its reference, only the movement of these areas are required to be encoded.
Therefore, the frame size of a P-frame is smaller than an I-frame since the redundant data is removed after the encoding procedure.
However, the receiver should refer to the reference frame when decoding a P-frame and it cannot be decoded if the preceding reference frame is missed at the receiver side.
\item \textbf{B-frame}:
A B-frame is a video frame that is able to reference from both a preceding reference frame as well as a future reference frame.
Therefore, encoding video frames as B-frames can improve the encoding efficiency but will also increase the processing time and hence we do not consider the appearance of B-frames in this thesis.
\end{itemize}

We assume that all video frames are encoded as either I-frames or P-frames as shown in figure~\ref{fig::baselineGOP} (which is the baseline profile in H.264), where a P-frame will reference from its preceding frame (can be either an I-frame or a P-frame).
An I-frame will be repeatedly inserted after a give number of P-frames, and we denote an I-frame together with its following P-frames as a group of pictures (GOP) in this thesis.
In the following chapter, we will show how to apply our proposed correlated data gathering mechanism for reducing encoded bits of a GOP in the multi-camera networks.