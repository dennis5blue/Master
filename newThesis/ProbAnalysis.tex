\section{Problem Analysis}
%\subsection{Problem Analysis and Decomposition}
\label{sec::OverallProbAnalysis}
%
As we mentioned before, Problem~\eqref{eq::formulation} belongs to a nonlinear integer programming (NIP) problem.
In addition, the determination of $\mathbf{X}$ is highly related to $\mathbf{Z}$ due to the reason that choosing $\mathbf{X}$ is restricted by the value of $\mathbf{Z}$ as shown in equation~\eqref{eq::prevSet} and~\eqref{eq::xConstraintForNonoverheard}.
Therefore, in order to reduce the high complexity for solving Problem~\eqref{eq::formulation} directly, we decompose it into two sub-problems such that the determination of $\mathbf{X}$ and $\mathbf{Z}$ does not correlated with each other any more.

We now present ``why'' and ``how'' we make such decomposition.
Note that since we assume that a P-frame can only reference from one I-frame as equation~\eqref{eq::referenceOnlyIframe} shows.
According to equation~\eqref{eq::referenceOnlyIframe}, every P-frames will reference from one previous scheduled I-frame to reduce its encoded bits.
That is, as long as a portion of cameras are determined to encoded their images as I-frames, the remaining P-frame cameras can find their most correlated I-frame camera and reference from it in polynomial time, which means that the determination of transmission schedule can be reduced to selecting I-frame cameras.
Consequently, we can decompose Problem~\eqref{eq::formulation} into two sub-problems, including (a) the \emph{I-frame selection sub-problem} and (b) the \emph{P-frame attachment sub-problem}, and these two sub-problems will be solved sequentially.
That is, the solution of \emph{I-frame selection sub-problem} will become the input of the \emph{P-frame attachment sub-problem} as shown in Fig.~\ref{fig::subProbProcedure}, where the goal of the \emph{I-frame selection sub-problem} is to select some cameras to encode their images as I-frames so that the remaining P-frame cameras can encode their images by referencing from those selected I-frames.
%
\begin{figure}
\begin{center}
\includegraphics[width=0.95\columnwidth]{Figures/procedureOfSubproblems.pdf}
\caption{\label{fig::subProbProcedure}Procedure for sub-problems division}
\end{center}
\end{figure}
%

Note that since we assume that only I-frame can become the reference frame as illustrated in equation~\eqref{eq::referenceOnlyIframe}, the \emph{I-frame selection sub-problem} is the most important part for maximizing the performance of overhearing source coding.
Therefore, we aim to find the optimal solution of the \emph{I-frame selection sub-problem} by a proposed branch-and-bound approach as presented in Chapter~\ref{sec::proposedBBAlg}.
Besides, we also propose a heuristic approach based on graph approximation for the determination of the set of I-frame cameras in Chapter~\ref{sec::graphApprox}.
After the selection of I-frame cameras, the performance of our proposed data gathering scheme is almost determined.
Therefore, we solve the \emph{P-frame scheduling sub-problem} through a greedy approach as presented in Chapter~\ref{sec::PFrameScheduling}.
%
\subsection{Analysis of I-Frame Selection Sub-Problem}
\label{sec::iFrameSelectionSubProb}
In this chapter, we show how we derive Problem~\eqref{eq::formulation} into the \emph{I-frame selection sub-problem}, and we will also give a comprehensive analysis about the formulated \emph{I-frame selection sub-problem}.
%
\subsubsection{Formulation for I-Frame Selection Sub-Problem}
\label{sec::iFrameSelectionSubProbFormulation}
Note that the goal of \emph{I-frame selection sub-problem} is to select some cameras to act as I-frame cameras, therefore, the decision variables of this sub-problem are the diagonal elements of $\mathbf{X}$.
In this thesis, for the sake of notation simplicity, we thus define
\begin{equation}
diag(\mathbf{X}) = \left[ x_{11}, x_{22}, \cdots, x_{|V||V|} \right],
\end{equation}
as a $1 \times |V|$ vector to represent the diagonal elements of $\mathbf{X}$.
Our goal is to figure out whether ${x_{ii},i \in V}$ equals to $0$ or $1$ so that the total amount of encoded bits can be minimized, where the other scheduling constraints (i.e. constraints about $\mathbf{Z}$) are ignore during the phase of selecting I-frame cameras for the sake of simplification.
Therefore, the simplified problem without scheduling constraints can be written as:
\begin{align}
\underset{\mathbf{X}}{\text{minimize}} & & \nonumber \\
	&\sum_{i=1}^{|V|} \sum_{j \in U_i}  x_{ij} h_{ij}, & \nonumber \\
\text{subject to} & & \nonumber \\
	&U_i = \{ j \in V | d_{ij} \leq \rho d_{j0} \}, &\forall i \in V, \nonumber \\
	&\sum_{j \in U_i} x_{ij} = 1, &\forall i \in V, \nonumber \\
	&x_{ij} \leq x_{jj}, &\forall i \in V, \forall j \in W_i, \nonumber \\
	&x_{ij} = 0, &\forall i \in V, \forall j \in V \setminus U_i, \nonumber \\
	&x_{ij} = \{0,1\}, &\forall i,j \in V.
\label{eq::formulationSimplified}
\end{align}
For the formulation in Problem~\eqref{eq::formulationSimplified}, the mathematical form is also complicated, therefore, we define a new variable $u_{ij}$ such that $u_{ij} = 1$ if $j \in U_i$, otherwise, $u_{ij} = 0$.
Note that the meanings about the following three constraints
\begin{align}
&\sum_{j \in U_i} x_{ij} = 1, &\forall i \in V, \nonumber \\
&x_{ij} \leq x_{jj}, &\forall i \in V, \forall j \in W_i, \nonumber \\
&x_{ij} = 0, &\forall i \in V, \forall j \in V \setminus U_i,
\label{eq::threeConstraintsToSimplfied}
\end{align}
is to ensure that only I-frame cameras can be referenced and each P-frame camera should reference from an overheard I-frame camera.
In order to further simply Problem~\eqref{eq::formulationSimplified}, we use a simple constraint to express the three constraints in~\eqref{eq::threeConstraintsToSimplfied}, which is written as:
\begin{equation}
\sum_{j=1}^{|V|} u_{ij} x_{jj} \geq 1, \forall i \in V.
\label{eq::moreThanOneCanHear}
\end{equation}
The interpretation of equation~\eqref{eq::moreThanOneCanHear} is that each P-frame camera must have at least one I-frame camera that it can overhear, and if a camera $i$ cannot overhear any other cameras (i.e. $u_{ij} = 0, \forall j \neq i$), then camera $i$ must be an I-frame camera (i.e. $x_{ii}=1$).
Using equation~\eqref{eq::moreThanOneCanHear} to represent the three constraints in~\eqref{eq::threeConstraintsToSimplfied}, Problem~\eqref{eq::formulationSimplified} is thus equivalent to the following formulation:
\begin{align}
\underset{\mathbf{X}}{\text{minimize}} & & \nonumber \\
	&\sum_{i=1}^{|V|} \sum_{j=1}^{|V|}  x_{ij} h_{ij}, & \nonumber \\
\text{subject to} & & \nonumber \\
	&\sum_{j=1}^{|V|} u_{ij} x_{jj} \geq 1, &\forall i \in V, \nonumber \\
	&x_{ij} = \{0,1\}, &\forall i,j \in V.
\label{eq::formulationSimplifiedAsGraph}
\end{align}

As we mentioned before, the goal of the \emph{I-frame selection sub-problem} is to determine the set of I-frame cameras.
Therefore, we only care about the diagonal elements of $\mathbf{X}$, which means that the variables $x_{ij}, i \neq j$ are not what we aim to solve in this sub-problem.
For this reason, we can further simplify Problem~\eqref{eq::formulationSimplifiedAsGraph} such that only $x_{ii}, \forall i \in V$ is considered in the formulation.
In order to achieve this goal, we note that if camera $i$ is encoded as an I-frame, it is clearly that it requires $h_{ii}$ to encode its image.
However, if camera $i$ is encoded as a P-frame, the amount of encoded bits should depend on which I-frame it references from.
An intuitive method to determine the reference frame is to select the most correlated I-frame, and the amount of encoded bits can be written as $\underset{u_{ij}x_{jj} = 1}{\min} h_{ij}$, where $u_{ij}x_{jj} = 1$ indicates the I-frame camera that camera $i$ can overhear.
Therefore, if we denote ${\hat{h}_i = \underset{u_{ij}x_{jj} = 1}{\min} h_{ij}}$ as the smallest amount of encoded bits if camera $i$ is a P-frame camera, Problem~\eqref{eq::formulationSimplifiedAsGraph} can be transformed into:
\begin{align}
\underset{x_{ii},i \in V}{\text{minimize}} & & \nonumber \\
	&\sum_{i=1}^{|V|} x_{ii}h_{ii} + \sum_{i=1}^{|V|} (1-x_{ii}) \hat{h}_i, & \nonumber \\
\text{subject to} & & \nonumber \\
	&\sum_{j=1}^{|V|} u_{ij} x_{jj} \geq 1, &\forall i \in V, \nonumber \\
	&x_{jj} = \{0,1\}, &\forall j \in V.  
\label{eq::IframeSelectionSubProblem}
\end{align}
Problem~\eqref{eq::IframeSelectionSubProblem} describes the decomposed \emph{I-frame selection sub-problem}, we will further give an analysis of this problem and then show our proposed algorithm for solving it in the followings.

%
\subsubsection{Analysis for I-Frame Selection Sub-Problem}
In this chapter, we will give a comprehensive analysis of Problem~\eqref{eq::IframeSelectionSubProblem} formulated in Chapter~\ref{sec::iFrameSelectionSubProbFormulation}.
First, we will prove the $\mathcal{NP}$-hardness of Problem~\eqref{eq::IframeSelectionSubProblem}, and then we will point out some possible directions for solving this problem.
To start proving the $\mathcal{NP}$-hardness, we first give a definition of the well-known \emph{set covering problem} as:
\begin{mydef}[\textbf{Set Covering Problem}]
The set covering problem is a $\mathcal{NP}$-complete problem~\cite{SetCoveringNPComplete} such that given a set of elements $V = \{1,2,\cdots,|V|\}$ and a set $S$ of $n$ sets whose union equals $V$, the goal of the set covering problem is to identify the smallest subset of $S$ whose union equals $V$.
\label{def::SCP}
\end{mydef}
Note that the set covering problem can be formulated in the form of an integer programming problem~\cite{SetCoveringFormulation} as:
\begin{align}
\underset{\mathbf{X}}{\text{minimize}} & & \nonumber \\
 &\sum_{j \in S} c_j x_j, & \nonumber \\
\text{subject to} & & \nonumber \\
 &\sum_{j \in S} a_{ij}x_j \geq 1, &\forall i \in V, \nonumber \\
 &x_{j} = \{0,1\}, &\forall j \in S,
\label{eq::SetCoveringProblemFormualtion}
\end{align}
where $c_j > 0$ is the coverage cost of set $j$, $x_j$ is a binary decision variable indicates that if set $j$ is selected, and $a_{ij}$ is a binary parameter shows that whether node $i$ is covered by the set $j$.
\begin{mylem}
Considering a special case of Problem~\eqref{eq::IframeSelectionSubProblem}, this problem is equivalent to a set covering problem if the cost of P-frame cameras is small enough to be neglected.
\label{lemma::ReduceProb}
\end{mylem}
\textbf{\emph{Proof:}}
To start the proof, we first model the transmission range of each camera into sets.
That is, the set $S$ will contain $|V|$ different sets, where all the members of the $j^{th}$ set are those cameras which can overhear the transmission of camera $j$ (including camera $j$).
More specifically, if $u_{ij} = 1$ then it means that $i$ is a member of the $j^{th}$ set.
Therefore, the goal for selecting I-frame cameras such that all P-frame cameras can have at least one I-frame camera to reference thus become selecting sets such that the union of those selected sets equalled to the set of all cameras.
To proceed, note that we make the assumption in lemma~\ref{lemma::ReduceProb} that the cost of P-frame cameras is small enough to be neglected.
Therefore, the remaining part for the objective value in Problem~\eqref{eq::IframeSelectionSubProblem} is the cost for selecting I-frame cameras, which is equivalent to the coverage cost of $j$, and hence we finish showing that the special case of Problem~\eqref{eq::IframeSelectionSubProblem} is equivalent to the set covering problem.\hfill$\mathsmaller{\mathsmaller{\blacksquare}}$

\begin{mythm}
For the I-frame selection sub-problem~\eqref{eq::IframeSelectionSubProblem} with $u_{ij}, \forall i,j$ given, finding the optimal solution is $\mathcal{NP}$-hard.
\end{mythm}
\textbf{\emph{Proof:}}
According to definition~\ref{def::SCP} and lemma~\ref{lemma::ReduceProb}, we can show that the special case of the reduced I-frame selection sub-problem is $\mathcal{NP}$-complete.
Therefore, by restriction~\cite{Restriction}, the original problem is $\mathcal{NP}$-hard.\hfill$\mathsmaller{\mathsmaller{\blacksquare}}$

Due to the $\mathcal{NP}$-hardness of Problem~\eqref{eq::IframeSelectionSubProblem}, we can claim that although the optimal solution can be found if we search through the feasible set, directly traversing the whole solution set would lead to very high computational complexity.
Therefore, some policy must be included for reducing the complexity.
Since all the decision variables ${x_{jj},j \in V}$ are binary integers, Problem~\eqref{eq::IframeSelectionSubProblem} belongs to a \emph{binary integer programming problem}, and one possible method for solving binary integer programming problem without losing the optimality is through the branch-and-bound (BB) algorithm~\cite{BB} as presented in Chapter~\ref{sec::proposedBBAlg}.
However, sometimes the applications for multi-camera networks does not concern the optimality due to real-time requirements.
We thus propose another algorithm to solve Problem~\eqref{eq::IframeSelectionSubProblem} in a more time efficient way by modeling it into a graph problem, where the proposed method will be described in Chapter~\ref{sec::graphApprox}.
%