\subsubsection{Inter Prediction in H.264}
\label{sec::MEIntro}
%
The idea of \emph{inter prediction} is to predict a block of luma and chroma samples from a reference frame that has previously been coded and transmitted.
More specifically, the encoder will select a prediction region, generate a prediction block and subtract this from the target encoding block to form a residual that is then coded and transmitted, where the offset between the position of the current encoding block and the prediction region in the reference frame is denoted as a motion vector.
The process in H.264 for generating motion vector is called motion estimation, which is also known as one of the most computational expensive part for encoding video frames.
One possible techniques for motion estimation in H.264 is the block-matching algorithm as illustrated in Fig.~\ref{fig::MEIntro}, which will estimate the motion on the basis of rectangular blocks (e.g. macroblocks) and generate one motion vector for each block.
%
\begin{figure}
\begin{center}
\includegraphics[width=0.95\columnwidth]{Figures/motionEstimation.pdf}
\caption{\label{fig::MEIntro}Demonstration of motion estimation in H.264}
\end{center}
\end{figure}
%

For an encoder using block-matching algorithm, each frame will be divided into blocks, and then a search region will be defined for finding the most matched block from a reference frame.
As long as the most matched block is found, its position is recorded using a motion vector.
Afterwards, the encoding block is subtracted from the prediction of the reference frame.
As a Consequence, the motion vector and the resulting subtracted difference can be transmitted instead of the original encoding block, and hence inter-frame redundancy has been removed and data compression is achieved.
At the receiver side, a H.264 decoder will reconstruct the block by adding the difference signal to the most matched block of the reference frame.