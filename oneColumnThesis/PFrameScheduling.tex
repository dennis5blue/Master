\subsection{Decomposed P-Frame Association Sub-Problem}
\label{sec::PFrameScheduling}
As Fig.~\ref{fig::subProbProcedure} shows, the input for the \emph{P-frame association sub-problem} is the solution of the \emph{I-frame selection sub-problem}.
Besides, we also note that after Problem~\eqref{eq::IframeSelectionSubProblem} is solved, all the elements in $diag(\mathbf{X})$ are determined.
If we denote ${diag^*(\mathbf{X}) = \left[ x_{11}^*, \cdots, x_{|V||V|}^* \right]}$ as the solution of the \emph{I-frame selection sub-problem}, then the set of determined I-frame cameras can be well defined as:
\begin{equation}
\mathcal{I}^* = \{ i \in V | x_{ii}^* = 1\},
\label{eq::solIFrameCams}
\end{equation}
and the set of remaining P-frame cameras will be
\begin{equation}
\mathcal{P}^* = V \setminus \mathcal{I}^*.
\label{eq::solPFrameCams}
\end{equation} 
%
\begin{mylem}
By the solution vector of I-frame selection sub-problem $diag^*(\mathbf{X})$, the two sets $\mathcal{I}^*$ and $\mathcal{P}^*$ can be well defined, and the best reference I-frame camera in $\mathcal{I}^*$ of all remaining P-frame cameras in $\mathcal{P}^*$ can be found in linear time.
\label{lemma::findBestICamEasy}
\end{mylem}
\textbf{\emph{Proof:}}
Obviously, $\mathcal{I}^*$ and $\mathcal{P}^*$ can be obtained directly by equation~\eqref{eq::solIFrameCams} and~\eqref{eq::solPFrameCams}, and hence we only need to prove the best reference I-frame camera of all remaining P-frame cameras can be found in linear time.
Note that as Problem~\eqref{eq::IframeSelectionSubProblem} shows, ${\hat{h}_i = \underset{u_{ij}x_{jj} = 1}{\min} h_{ij}}$, is the smallest amount of encoded bits for camera $i$ if $i \in \mathcal{P}^*$.
Therefore, we can claim that if $h_{ik} = \hat{h}_i$, then $k$ is the best reference I-frame camera of camera $i$.
In this way, finding the best reference I-frame camera of all remaining P-frame cameras is equivalent to solving the minimization problem ${\underset{u_{ij}x_{jj} = 1}{\min} h_{ij}}$, with ${u_{ij}, i,j \in V}$ and ${x_{jj}, j \in V}$ given, and this can be further understood as finding the smallest value in an unsorted array whose time complexity is $\mathcal{O}(N)$.
Consequently, it can be solved in linear time.\hfill$\mathsmaller{\mathsmaller{\blacksquare}}$

\begin{mythm}
The determination of $diag(\mathbf{X})$ is equivalent to the determination of all elements in $\mathbf{X}$, which means that $\mathbf{X}$ can be obtained uniquely through the solution of I-frame selection sub-problem $diag^*(\mathbf{X})$.
\label{theorem::determineX}
\end{mythm}
\textbf{\emph{Proof:}}
For camera $i$, if $x_{ii}^* =1$, it means that camera $i$ is an I-frame camera and hence by constraint~\eqref{eq::referenceConstraint} and~\eqref{eq::xConstraintForNonoverheard}, $x_{im}=0, \forall m \neq i$, which means that the $i^{th}$ row of $\mathbf{X}$ is determined.
For camera $j$, if $x_{jj}^* =0$, it means that camera $j$ is a P-frame camera.
By lemma~\ref{lemma::findBestICamEasy}, we know that with the help of $diag^*(\mathbf{X})$, the best reference I-frame camera in $\mathcal{I}^*$ of all remaining P-frame cameras in $\mathcal{P}^*$ can be found in linear time, where we assume that the best reference I-frame camera for camera $j$ is camera $k$.
In this way, $x_{jk}=1$, and by constraint~\eqref{eq::referenceConstraint} and~\eqref{eq::xConstraintForNonoverheard}, we know that $x_{jm}=0, \forall m \neq k$, which means that the $j^{th}$ row of $\mathbf{X}$ is also determined.
As a consequence, both the rows for camera in $\mathcal{I}^*$ and $\mathcal{P}^*$ are all determined, and hence all elements in $\mathbf{X}$ can be determined by $diag^*(\mathbf{X})$ due to the reason that $\mathcal{I}^* \cup \mathcal{P}^* = V$.\hfill$\mathsmaller{\mathsmaller{\blacksquare}}$

According to theorem~\ref{theorem::determineX}, we know that $\mathbf{X}$ can be uniquely determined after the \emph{I-frame selection sub-problem} is solved.
Therefore, the remaining unsolved decision variable in the original Problem~\eqref{eq::formulation} is $\mathbf{Z}$.
However, note that $\mathbf{Z}$ has not been considered in the objective function of Problem~\eqref{eq::formulation}.
Moreover, the objective value of Problem~\eqref{eq::formulation} is only affected by $\mathbf{X}$, which means that Problem~\eqref{eq::formulation} can be transform from an ``optimization problem'' into a ``decision problem'' after $\mathbf{X}$ is determined.
For this reason, we consider the \emph{P-frame association sub-problem} as a decision problem such that we only need to check whether $\mathbf{X}$ obtained by the \emph{I-frame selection sub-problem} is feasible under the following constraints:
\begin{align}
	&U_i = \{ j \in V | d_{ij} \leq \rho d_{j0} \}, &\forall i \in V, \nonumber \\
	&W_i = \{ j \in U_i | z_j \leq z_i \}, &\forall i \in V, \nonumber \\
	&\sum_{j \in W_i} x_{ij} = 1, &\forall i \in V, \nonumber \\
	&|z_i - z_j| > 0, &\forall i,j \in V, i \neq j, \nonumber \\
	&1 \leq z_i \leq N, &\forall i \in V, z_i \in \mathbb{Z}. \nonumber \\
\label{eq::PFrameSchedulingSubProb}
\end{align}
As long as constraints~\eqref{eq::PFrameSchedulingSubProb} are satisfied, we can claim that Problem~\eqref{eq::formulation} is solved, and its objective value can be calculated directly through $\mathbf{X}$.
%
\begin{mythm}
If $diag^*(\mathbf{X})$ is a feasible solution of Problem~\eqref{eq::IframeSelectionSubProblem}, and $\mathbf{X}$ is generated by $diag^*(\mathbf{X})$ according to theorem~\ref{theorem::determineX}, then constraints~\eqref{eq::PFrameSchedulingSubProb} can always be satisfied by adjusting proper value of $\mathbf{Z}$.
\label{theorem::IFrameSolalwaysFeasible}
\end{mythm}
\textbf{\emph{Proof:}}
From theorem~\ref{theorem::determineX}, the following two statements of elements in $\mathbf{X}$ are already been proved to be true:
\begin{enumerate}
\item For I-frame camera $i$, $x_{ii}=1$ and ${x_{im} = 0}, {\forall m \neq i}$, where we also know that $i \in U_i$ from equation~\eqref{eq::inTransRangeSimple}.
Therefore, for I-frame camera $i$, ${\underset{m \in U_i}{\sum} x_{im}=1},{\forall i \in \mathcal{I}^*}$ is satisfied.
\item For P-frame camera $j$, we can find a camera $k$, ${k \in U_j}$ in linear time such that ${x_{jk} = 1}$ and ${x_{jm}=0}, \forall m \neq k$.
Therefore, for P-frame camera $j$, ${\underset{m \in U_j}{\sum} x_{jm}=1},{\forall j \in \mathcal{P}^*}$ is also satisfied.
\end{enumerate}
According to the above two statements, since $\mathcal{I}^* \cup \mathcal{P}^* = V$, we can thus claim that ${\underset{j \in U_i}{\sum} x_{ij}=1},{\forall i \in V}$.
To proceed, in order to show constraints~\eqref{eq::PFrameSchedulingSubProb} are satisfied, now we only need to prove ${\underset{j \in U_i}{\sum} x_{ij} = \underset{j \in W_i}{\sum} x_{ij}},{\forall i \in V}$ is true by adjusting proper $\mathbf{Z}$.
Note that there might exist various $\mathbf{Z}$ such that ${\underset{j \in U_i}{\sum} x_{ij} = \underset{j \in W_i}{\sum} x_{ij}}$, however, we only need to show one possible method for adjusting $\mathbf{Z}$ and the proof can be done.
Note that $W_i \subseteq U_i$, which means that ${\underset{j \in U_i}{\sum} x_{ij} \geq \underset{j \in W_i}{\sum} x_{ij}}$.
Besides, we have shown that ${\underset{j \in U_i}{\sum} x_{ij}=1}$, therefore, if we can prove that ${k = \underset{j \in U_i}{\arg \max} \quad x_{ij}}$ is also contained in the set $W_i$, then we can claim that ${\underset{j \in U_i}{\sum} x_{ij} = \underset{j \in W_i}{\sum} x_{ij} = 1}$.
To start, we first calculate the set of I-frame cameras $\mathcal{I}^*$ and P-frame cameras $\mathcal{P}^*$ by equation~\eqref{eq::solIFrameCams} and~\eqref{eq::solPFrameCams}.
For camera $i \in \mathcal{I}^*$, according to lemma~\ref{lemma::findBestICamEasy}, each P-frame camera is able to find its best reference I-frame camera, which also means that camera $i$ can know the set of P-frame cameras that reference from it.
If we denote ${D_i= \{j \in \mathcal{P}^*|x_{ji} = 1\}},{i \in \mathcal{I}^*}$ as the set of P-frame cameras that reference from I-frame camera $i$, then ${j \in \{U_i \cap W_i\}},{\forall j \in D_i}$ as long as ${z_i < z_j}, {\forall i \in \mathcal{I}^*},{\forall j \in D_i}$, this is thus equivalent to showing that ${\underset{j \in U_i}{\arg \max} \quad x_{ij} \in W_i}$.
Consequently, the proof is done, and the policy for adjusting $\mathbf{Z}$ is to ensure ${z_i < z_j}, {\forall i \in \mathcal{I}^*},{\forall j \in D_i}$, which can be achieved by scheduling I-frame camera first, followed by its listener P-frame cameras one by one.
\hfill$\mathsmaller{\mathsmaller{\blacksquare}}$
%
\begin{mythm}
If $diag^*(\mathbf{X})$ is an optimal solution for Problem~\eqref{eq::IframeSelectionSubProblem}, then its relative $\mathbf{X}$ is also optimal for Problem~\eqref{eq::formulation}.
\label{theorem::optimalIsOptimal}
\end{mythm}
\textbf{\emph{Proof:}}
According to theorem~\ref{theorem::determineX}, $\mathbf{X}$ can be determined uniquely by $diag^*(\mathbf{X})$.
Since the objective functions of Problem~\eqref{eq::IframeSelectionSubProblem} and Problem~\eqref{eq::formulation} are equivalent, if $diag^*(\mathbf{X})$ is optimal for Problem~\eqref{eq::IframeSelectionSubProblem}, then it relative $\mathbf{X}$ can lead to the smallest objective value of Problem~\eqref{eq::formulation}.
Besides, by theorem~\ref{theorem::IFrameSolalwaysFeasible}, we know that $\mathbf{X}$ is feasible for Problem~\eqref{eq::formulation}, and hence it is optimal.
\hfill$\mathsmaller{\mathsmaller{\blacksquare}}$

By theorem~\ref{theorem::optimalIsOptimal}, we can learn that letting ${z_i < z_j}$, ${\forall i \in \mathcal{I}^*}$, ${\forall j \in D_i}$ can make $\mathbf{X}$ and $\mathbf{Z}$ become the optimal solution of Problem~\eqref{eq::formulation} as long as $diag^*(\mathbf{X})$ (i.e. $\mathcal{I}^*$ and $\mathcal{P}^*$) is optimal for Problem~\eqref{eq::IframeSelectionSubProblem}.
One possible approach to ensure ${z_i < z_j}$, ${\forall i \in \mathcal{I}^*}$, ${\forall j \in D_i}$ is to allocate transmission time slots group by group, where a group is the cameras set of an I-frame camera and its listeners (i.e. $\{i \cup D_i\},{\forall i \in \mathcal{I}^*}$).
Within the transmission of one group, the I-frame camera should be scheduled at the beginning, where the order for remaining P-frame cameras does not influence the objective value of Problem~\eqref{eq::formulation}.
Therefore, in this thesis, we just determine the schedule of P-frame cameras based on a greedy approach, that is, a camera that can make more other P-frames cameras to be able to overhear its transmission will be scheduled first.
The reason why we use this metric is that the later scheduled P-frame cameras might have the possibility to reference from the previous P-frame camera instead of the first scheduled I-frame camera.
However, referencing from a P-frame camera will increase the complexity of the scheduling problem and hence it is leave as a future work of this thesis.
Besides, we also note that the greedy P-frame scheduling metric can be changed for fitting into any other application requirements (e.g. camera which is more delay sensitive to be scheduled first) since changing the schedule of P-frame cameras does not cause a performance loss of Problem~\eqref{eq::formulation}.

\begin{figure}
\begin{center}
\includegraphics[width=0.95\columnwidth]{Figures/PscheOverview.pdf}
\caption{\label{fig::pScheOverview}Demonstration of transmission schedule}
\end{center}
\end{figure}
%
Fig.~\ref{fig::pScheOverview} shows an example of the determined transmission schedule.
Suppose that camera $4$ and $7$ are selected to be I-frame cameras (i.e.  ${\mathcal{I}^*=\{4,7\}}$) by algorithm~\ref{alg::bbAlgorithm} or~\ref{alg::minWeightDS}, and cameras $1,2,3$ are more correlated with camera $4$ whereas cameras $5,6$ are more correlated with camera $7$.
Therefore, cameras $1,2,3$ will join the group of camera $4$ while cameras $5,6$ will join the group of camera $7$ (i.e. ${D_4=\{1,2,3\}}$ and ${D_7=\{5,6\}}$).
Since time slots are allocated group by group, assume that the group of camera $4$ is scheduled first, then slot $1$ will be allocated for camera $4$ since I-frame camera should be scheduled at the beginning so that the later P-frame cameras are able to overhear its transmission.
To proceed, time slots $2,3,4$ will be allocated for the member cameras of camera $4$ (i.e. cameras in $D_4$), and which member camera is scheduled at which time slot is based on the greedy P-frame scheduling metric.
After all member cameras of camera $4$ is scheduled, the next group will continued to be scheduled.
That is, slot $5$ is for the transmission of camera $7$ (the I-frame camera of the next group), and slots $6,7$ are for the member cameras of camera $7$ (i.e. cameras in $D_7$).
After all cameras groups are scheduled (i.e. ${\{4,D_4\}}$ and ${\{7,D_7\}}$), the procedure for P-frame association is done.