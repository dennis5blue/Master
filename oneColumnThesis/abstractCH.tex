多重相機網路的一大特色是資料量龐大,並且有低傳輸延遲的需求。因此,不同於傳統網路下考慮如何最大化個別相機的傳輸效能,我們以如何能更有效的提升頻譜使用效率作為出發點,試圖在不影響影像蒐集品質下最小化實際所需傳輸的資料量。基於鄰近相機之間所拍攝到的影像可能會有所重疊,我們引入了多視角視訊編碼技術來透過重疊影像之相關性以降低影像編碼所需的資料量。在本篇論文中,我們提出了一個利用側聽編碼來去除不同相機所攝影像間的相關資料的技術,此一技術的優點是相機不需要在傳輸前彼此交換資訊,也因此並不會造成頻譜的額外負擔。為了使側聽編碼的效能最大化,我們首先提出了一個綜合考量參考圖像以及相機排程的最佳化問題,並且在不損失最佳化特性的前提下將此問題拆解成「I-圖像相機選擇」以及「P-圖像相機參考圖像決定」的這兩個問題,使得P-圖像能夠選擇與其相關性最大的I-圖像作為移動向量估計時之參考圖像。我們接著提出了兩種演算法來解決I-圖像相機選擇的問題,其一是藉由分支界定法以得到此問題的最佳解,其二則是透過圖學近似的方法來更有效率的求得一近似最佳值的解。此外,在實際多重相機網路的應用當中,相機之間的相關性會隨著不同時間下拍攝到的影像而有所變化。因此我們也提出了何時應該重新選擇I-圖像的準則,使得整體的網路能夠在不增加太多計算複雜度的前提下,得以迎合不同時間點的相關性變化。透過3D建模軟體當中虛擬城市拍攝到的影像以及電腦模擬結果的分析,我們驗證了側聽編碼的技術能夠確實的增加頻譜使用上的效率,也因此確立了側聽編碼確實為一可能運用在缺乏頻譜資源時的多重相機網路下之資料蒐集技術。