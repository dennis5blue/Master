\subsection{Graph Approximation for I-Frame Selection Problem}
\label{sec::graphApprox}
In addition to finding the optimal solution of Problem~\eqref{eq::IframeSelectionSubProblem} through a BB approach, we are also wondering if a near optimal solution can be obtained under lower computational complexity.
One possible approach to reduce complexity can be done by approximating the problem into a graph problem.
For this reason, we will introduce in the followings how we transform problem~\eqref{eq::IframeSelectionSubProblem} into a \emph{minimum weighted dominating set} problem and solve it through a heuristic algorithm.
%
\subsubsection{Minimum Weighted Dominating Set}
To start, we model the multi-camera network into a directed graph $\mathcal{G}=(V,E)$, where $V$ is the set of all vertices (same as the set of all cameras) and there exists a directed edge $(i,j) \in E$ (note that the edge $(i,j)$ is considered to be directed from $i$ to $j$) if and only if camera $i$ has the possibility for overhearing the transmission of camera $j$ (i.e. $j \in U_i$~\eqref{eq::inTransRange}).
For the sake of simplicity, we assume that the vertex index of $\mathcal{G}$ is the same with the camera index, that is, vertex $i \in V$ represents camera $i$ and it aims to transmit $F_i$ to the aggregator.
We then denote $\mathcal{N}_i$ as the set of all neighbors of a vertex $i \in V$ such that $j \in \mathcal{N}_i$ if and only if $(i,j) \in E$.
Since our goal in problem~\eqref{eq::IframeSelectionSubProblem} is to minimize the total cost (encoded bits) of the multi-camera network, $\mathcal{G}$ is considered as a weighted graph having both vertices and edges being weighted, where the weight for a vertex $i \in V$ is defined as:
\begin{equation}
\omega_i = h_{ii},
\end{equation}
whereas the weight for an edge $(i,j) \in E$ is:
\begin{equation}
\omega_{i,j} = h_{ij}.
\end{equation}
Note that it is not required $\omega_{i,j} = \omega_{j,i}$ due to the reason that $h_{ij}$ might not equaled to $h_{ji}$.
As an example, the directed graph $\mathcal{G}$ of a simple multi-camera network is shown in Fig.~\ref{fig::mdsGraphExample}.
%
\begin{figure}
\begin{center}
\includegraphics[width=0.95\columnwidth]{Figures/mdsGraphExample.pdf}
\caption{\label{fig::mdsGraphExample}A simple multi-camera network and its correlated graph}
\end{center}
\end{figure}
%

To proceed, the determination of I-frame cameras can be regarded as partitioning $\mathcal{G}$ into several groups such that each group has exactly one I-frame cameras.
The other members in the camera group is considered to be P-frame cameras and they will reference from the I-frame camera in their group.
In order to meet the requirement that all cameras must encode its frame as either an I-frame or a P-frame, the following properties are needed to be satisfied when forming camera groups:
\begin{itemize}
\item Each P-frame cameras must have at least one I-frame camera as its neighbor.
\item If a camera has no neighbor, it must be an I-frame camera.
\item If a camera has multiple I-frame cameras as its neighbor, it will join the group having the smallest cost of edge.
\end{itemize}
The first two properties can also be regarded as the dominance properties, while the third property is for minimizing the weight of the grouping structure.
Therefore, the problem for forming camera groups is similar to finding a minimum weighted dominating set of a directed graph.
The authors in~\cite{MWDS_baseline} also investigated the problem of finding a minimum weighted dominating set, however, the weight of edges (i.e. cost for P-frame cameras) is not considered in their work and hence their proposed algorithm will lead to sub-optimal solution in our target scenario.
For this reason, we take both vertex weight and edge weight into consideration during the formation of camera groups, where our proposed algorithm is presented in the following chapter.
%
\subsubsection{Proposed Cameras Grouping Algorithm}
Before introducing our proposed algorithm, we first note that there has a trade-off for a camera $i$ to determine whether it should be an I-frame camera or not.
On the one hand, if camera $i$ is assigned to be an I-frame camera, it will have the possibility to reduce the cost of cameras having camera $i$ as their neighbors.
That is, the advantage for selecting camera $i$ to be an I-frame camera can be written as:
\begin{equation}
\sum_{j:i \in \mathcal{N}_j} \left( \omega_j - \omega_{j,i} \right),
\label{eq::advForSelectAsI}
\end{equation}
where we only sum up $j$ such that $i \in \mathcal{N}_j$ in equation~\eqref{eq::advForSelectAsI} is because camera $i$ can only benefit the cameras fall in its transmission range.
On the other hand, camera $i$ cannot reduce its encoded bits as long as it is assigned to be an I-frame camera, therefore, the price for choosing camera $i$ as an I-frame camera is
\begin{equation}
- \omega_i.
\label{eq::costForSelectAsI}
\end{equation}
Our I-frame selection metric is thus defined as the summation of equation~\eqref{eq::advForSelectAsI} and~\eqref{eq::costForSelectAsI}, that is, the I-frame selection metric for camera $i$ can be written as:
\begin{equation}
\mu_i = - \omega_i + \sum_{j:i \in \mathcal{N}_j} \left( \omega_j - \omega_{j,i} \right),
\label{eq::IFrameSelectionMetric}
\end{equation}
where the larger $\mu_i$ is, the more possible that camera $i$ is selected as an I-frame camera.

We use that notation that $\mathcal{I}_{MWDS}$ and $\mathcal{P}_{MWDS}$ being the set of I-frame cameras and P-frame cameras determined by the minimum weighted dominating set problem, respectively.
Start with $\mathcal{I}_{MWDS} = \emptyset$ and $\mathcal{P}_{MWDS} = \emptyset$, our proposed algorithm then goes as follows.
During each iterations, the camera with the highest I-frame selection metric among all undetermined cameras will become an I-frame camera.
That is, camera $k$ is added into the I-frame cameras set $\mathcal{I}_{MWDS}$ such that
\begin{equation}
\mathcal{I}_{MWDS} = \mathcal{I}_{MWDS} \cup \{ k \},
\label{eq::updateICamsSet}
\end{equation}
where
\begin{equation}
k = \underset{i \in V \setminus \{ \mathcal{I}_{MWDS} \cup \mathcal{P}_{MWDS} \} }{\arg \max} \mu_i.
\end{equation}

After the insertion of I-frame cameras set, we then check if camera $k$ is the \emph{best} reference camera for those undetermined cameras, if true, those cameras with camera $k$ being its best reference camera will join camera $k$'s group; otherwise, the encoding style still cannot be determined.
Note that we assume that the frame of a P-frame camera cannot be referenced, therefore, as long as a camera has been decided to join the group of an I-frame camera, its image can no longer become a reference frame.
Consequently, P-frame cameras are not considered when judging each camera's \emph{best} reference camera, and hence if we denote $g_k$ as the cameras group for the I-frame camera $k$, an undetermined camera $i$ will join $g_k$ if and only if:
\begin{equation}
k = \underset{j \in V \setminus \mathcal{P}_{MWDS}}{\arg \min} \omega_{i,j}.
\label{eq::cameraGroupJoinCriteria}
\end{equation}
Based on equation~\eqref{eq::cameraGroupJoinCriteria}, some cameras might be selected for adding to $g_k$, afterwards, the overall P-frame cameras set will be updated as
\begin{equation}
\mathcal{P}_{MWDS} = \mathcal{P}_{MWDS} \cup \{ g_k \setminus \{k\} \}.
\label{eq::updatePCamsSet}
\end{equation}
In short, we summarize our proposed grouping algorithm for constructing a minimum weighted dominating set in algorithm~\ref{alg::minWeightDS}.
%
\IncMargin{1em}
\begin{algorithm}[]
 \SetAlgoLined
 \SetKwInOut{Input}{Input}\SetKwInOut{Output}{Output}
 \Input{Directed graph $\mathcal{G}=(V,E)$, and the weight for vertices $\omega_i$ and edges $\omega_{i,j}$}
 \Output{A minimum weighted dominating set of $\mathcal{G}$}
 \BlankLine
 Initialize $\mathcal{I}_{MWDS} \gets \emptyset$ and $\mathcal{P}_{MWDS} \gets \emptyset$ \\
 \While{$V \setminus \{ \mathcal{I}_{MWDS} \cup \mathcal{P}_{MWDS} \} \neq \emptyset $}
 {
 	\For{$i \in V \setminus \{ \mathcal{I}_{MWDS} \cup \mathcal{P}_{MWDS} \}$}
 	{
 		Calculate the I-frame selection metric $\mu_i$ by equation~\eqref{eq::IFrameSelectionMetric} \\
 	}
 	Select one camera $k$ to become an I-frame camera, where $k \gets \underset{i \in V \setminus \{ \mathcal{I}_{MWDS} \cup \mathcal{P}_{MWDS} \} }{\arg \max} \mu_i$ \\
 	Update the overall I-frame cameras set as $\mathcal{I}_{MWDS} \gets \mathcal{I}_{MWDS} \cup \{ k \}$ \\
 	Initialize the cameras group for camera $k$ as $g_k \gets \{ k \}$ \\
 	\For{$i \in V \setminus \{ \mathcal{I}_{MWDS} \cup \mathcal{P}_{MWDS} \}$}
 	{
 		\If{$ \omega_{i,k} \leq \underset{j \in V \setminus \mathcal{P}_{MWDS}}{\min} \omega_{i,j}$}
 		{
 			Add camera $i$ to the group of I-frame camera $k$, that is $g_k \gets g_k \cup \{ i \}$ \\
 			Update the overall P-frame cameras set as $\mathcal{P}_{MWDS} \gets \mathcal{P}_{MWDS} \cup \{ i \}$ \\
 		}
 	}
 }
 \caption{\label{alg::minWeightDS}Proposed algorithm for constructing minimum weight dominating set}
\end{algorithm}
\DecMargin{1em}
%
%
\begin{table*}[htb]
\footnotesize
\centering
\begin{tabularx}{0.9\textwidth}{c|c|X}
  \hline
  Type &Notation &Description \\
  \hline
  \hline
  \multirow{6}{*}
  {Sets} &$V$ &  Set of all vertices, which is also equalled to the set of all cameras\\
  	&$E$ & Set of all directed edges\\
  	&$\mathcal{N}_i$ & Set of all neighbors of camera $i$\\
  	&$\mathcal{I}_{MWDS}$ & Set of all I-frame cameras\\
  	&$\mathcal{P}_{MWDS}$ & Set of all P-frame cameras\\
  	&$g_k$ & Set of all members for the group of camera $k$, where camera $k$ must be an I-frame camera\\
  \hline
  \multirow{5}{*}
  {Parameters} &$\mathcal{G}$ & The directed graph generated by overhearing constraints\\
  	&$(i,j)$ & Directed edge from vertex $i$ to $j$, which indicated that camera $i$ can overhear the transmission of camera $j$\\
  	&$\omega_i$ & Weight of vertex $i$\\
  	&$\omega_{i,j}$ & Weight of edge $(i,j)$\\
  	&$\mu_i$ & I-frame selection metric for camera $i$\\
  \hline
\end{tabularx}
\\
\caption{\label{tab::MWDSSymbols}Additional notations introduced in proposed minimum weighted dominating set algorithm}
\end{table*}
%
%